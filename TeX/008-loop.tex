\hyperchapter{ch08}{ループ}{ループ}
\index{るぷ@ループ}

さて、ここまでで変数や関数、標準入出力といったプログラミングの基礎的な概念を教えてきた。あと1つでプログラミングに必要な基礎的な概念はすべて説明し終わる。ループだ。

\hypersection{ch0801}{これまでのおさらい}

C++では、プログラムは書いた順番に実行される。これを\texttt{逐次実行}\index{ちくじじつこう@逐次実行}という。

\begin{lstlisting}[language={C++}]
int main()
{
    std::cout << 1 ;
    std::cout << 2 ;
    std::cout << 3 ;
}
\end{lstlisting}

実行結果、

\begin{lstlisting}[style=terminal]
123
\end{lstlisting}

この実行結果が\,\texttt{"123"}\,以外の結果になることはない。C++ではプログラムは書かれた順番に実行されるからだ。

条件分岐\index{じようけんぶんき@条件分岐}は、プログラムの実行を条件付きで行うことができる。

\begin{lstlisting}[language={C++}]
int main()
{
    std::cout << 1 ;

    if ( false )
        std::cout << 2 ;

    std::cout << 3 ;

    if ( true )
        std::cout << 4 ;
    else
        std::cout << 5 ;
}
\end{lstlisting}

実行結果、

\begin{lstlisting}[style=terminal]
134
\end{lstlisting}

条件分岐によって、プログラムの一部を実行しないということが可能になる。

\hypersection{ch0802}{goto文}
\index{goto@\texttt{goto}文}\index{るぷ@ループ!goto@\texttt{goto}文}

ここでは繰り返し(ループ)の基礎的な仕組みを理解するために、最も原始的で最も使いづらい繰り返しの機能である\texttt{goto文}を学ぶ。\texttt{goto文}で実用的な繰り返し処理をするのは面倒だが、恐れることはない。より簡単な方法もすぐに説明するからだ。なぜ本書で\texttt{goto文}を先に教えるかというと、あらゆる繰り返しは、けっきょくのところ\texttt{if文}と\texttt{goto文}へのシンタックスシュガーにすぎないからだ。\texttt{goto文}を学ぶことにより、繰り返しを恐れることなく使う本物のプログラマーになれる。

\hypersubsection{ch080201}{無限ループ}
\index{むげんるぷ@無限ループ}\index{るぷ@ループ!むげん@無限〜}

\texttt{"hello{\textbackslash}n"}\,と3回出力するプログラムはどうやって書くのだろうか。\texttt{"hello{\textbackslash}n"}\,を1回出力するプログラムの書き方はすでにわかっているので、同じ文を3回書けばよい。

\begin{lstlisting}[language={C++}]
// 1回"hello\n"を出力する関数
void hello()
{
    std::cout << "hello\n"s ;
}

int main()
{
    hello() ;
    hello() ;
    hello() ;
}
\end{lstlisting}

10回出力する場合はどうするのだろう。10回書けばよい。コードは省略する。

では100回出力する場合はどうするのだろう。100回書くのだろうか。100回も同じコードを書くのはとても面倒だ。読者がVimのような優秀なテキストエディターを使っていない限り100回も同じコードを間違えずに書くことは不可能だろう。Vimならば1回書いたあとにノーマルモードで\,\texttt{"100."}\,するだけで100回書ける。

実際のところ、100回だろうが、1000回だろうが、あらかじめ回数がコンパイル時に決まっているのであれば、その回数だけ同じ処理を書くことで実現可能だ。

しかし、プログラムを外部から強制的に停止させるまで、無限に出力し続けるプログラムはどう書けばいいのだろうか。そういった停止しないプログラムを外部から強制的に停止させるには\texttt{Ctrl-C}を使う。

以下はそのようなプログラムの実行例だ。

\begin{lstlisting}[style=terminal]
$ make run
hello
hello
hello
hello
...
[Ctrl-Cを押す]
\end{lstlisting}

\texttt{goto文}は指定したラベルに実行を移す機能だ。

\begin{lstlisting}[style=grammar]
ラベル名 : 文

goto ラベル名 ;
\end{lstlisting}

\begin{lstlisting}[language={C++}]
int main()
{
    std::cout << 1 ;

    // ラベルskipまで飛ぶ
    goto skip ;

    std::cout << 2 ;

// ラベルskip
skip :
    std::cout << 3 ;
}
\end{lstlisting}

これを実行すると以下のようになる。

\ifTombow\enlargethispage{3mm}\fi
\begin{lstlisting}[style=terminal]
13
\end{lstlisting}

\texttt{2}を出力すべき文の実行が飛ばされていることがわかる。

これだけだと\,\texttt{"if (false)"}\,と同じように見えるが、\texttt{goto文}はソースコードの上に飛ぶこともできるのだ。

\begin{lstlisting}[language={C++}]
void hello()
{
    std::cout << "hello\n"s ;
}
int main()
{
loop :
    hello() ;
    goto loop ; 
}
\end{lstlisting}

これは\,\texttt{"hello{\textbackslash}n"}\,を無限に出力するプログラムだ。

このプログラムを実行すると、
\begin{enumerate}
\def\labelenumi{\arabic{enumi}.}
\item
  関数\texttt{hello}が呼ばれる
\item
  \texttt{goto}文でラベル\texttt{loop}まで飛ぶ
\item
  1.に戻る
\end{enumerate}
という処理を行う。

\hypersubsection{ch080202}{終了条件付きループ}
\index{しゆうりようじようけんつきるぷ@終了条件付きループ}\index{るぷ@ループ!しゆうりようじようけんつき@終了条件付き〜}

ひたすら同じ文字列を出力し続けるだけのプログラムというのも味気ない。もっと面白くてためになるプログラムを作ろう。例えば、ユーザーから入力された数値を合計し続けるプログラムはどうだろう。

いまから作るプログラムを実行すると以下のようになる。

\begin{lstlisting}[style=terminal]
$ make run
> 10
10
> 5
15
> 999
1014
> -234
780
\end{lstlisting}

このプログラムは、
\begin{enumerate}
\def\labelenumi{\arabic{enumi}.}
\item
  \texttt{">"}\,と表示してユーザーから整数値を入力
\item
  これまでの入力との合計値を出力
\item
  1.に戻る
\end{enumerate}
という動作を繰り返す。先ほど学んだ無限ループと同じだ。

さっそく作っていこう。

\begin{lstlisting}[language={C++}]
int input()
{
    std::cout << ">"s ;
    int x {} ;
    std::cin >> x ;
    return x ;
}

int main()
{
    int sum = 0 ;
loop :
    sum = sum + input() ;
    std::cout << sum << "\n"s ;
    goto loop ;
}
\end{lstlisting}

関数\texttt{input}は\,\texttt{">"}\,を表示してユーザーからの入力を得て戻り値として返すだけの関数だ。

\texttt{"sum = sum + input()"}\,は、変数\texttt{sum}に新しい値を代入するもので、その代入する値というのは、代入する前の変数\texttt{sum}の値と関数\texttt{input}の戻り値を足した値だ。

このような変数\texttt{x}に何らかの値\texttt{n}を足した結果を元の変数\texttt{x}に代入するという処理はとても多く使われるので、C++では\,\texttt{"x = x + n"}\,を意味する省略記法\,\texttt{"x += n"}\,がある。

\begin{lstlisting}[language={C++}]
int main()
{
    int x = 1 ;
    int n = 5 ;

    x = x + n ; // 6
    x += n ; // 11
}
\end{lstlisting}

さて、本題に戻ろう。上のプログラムは動く。しかし、プログラムを停止するには\texttt{Ctrl-C}を押すしかない。できればプログラム自ら終了してもらいたいものだ。

そこで、ユーザーが\texttt{0}を入力したときはプログラムを終了するようにしよう。

\begin{lstlisting}[language={C++}]
int input()
{
    std::cout << ">"s ;
    int x {} ;
    std::cin >> x ;
    return x ;
}

int main()
{
    int sum = 0 ;
loop :
    // 一度入力を変数に代入
    int x = input() ;
    // 変数xが0でない場合
    if ( x != 0 )
    {// 実行
        sum = sum + x ;
        std::cout << sum << "\n"s ;
        goto loop ;
    }
    // x == 0の場合、 ここに実行が移る
    // main関数の最後なのでプログラムが終了
}
\end{lstlisting}

うまくいった。このループは、ユーザーが\texttt{0}を入力した場合に繰り返しを終了する、条件付きのループだ。

\hypersubsection{ch080203}{インデックスループ}
\index{いんでつくするぷ@インデックスループ}\index{るぷ@ループ!いんでつくす@インデックス〜}

最後に紹介するループは、インデックスループだ。\(n\)回\,\texttt{"hello{\textbackslash}n"s}\,を出力するプログラムを書こう。問題は、この\(n\)はコンパイル時には与えられず、実行時にユーザーからの入力で与えられる。

\begin{lstlisting}[language={C++}]
// n回出力する関数の宣言
void hello_n( int n ) ;

int main()
{
    // ユーザーからの入力
    int n {} ;
    std::cin >> n ;
    // n回出力
    hello_n( n ) ;
}
\end{lstlisting}

このコードをコンパイルしようとするとエラーになる。これは実はコンパイルエラーではなくてリンクエラーという種類のエラーだ。その理由は、関数\texttt{hello\_n}に対する関数の定義が存在しないからだ。

関数というのは宣言と定義に分かれている。\index{かんすう@関数}\index{かんすう@関数!せんげん@宣言}\index{かんすう@関数!ていぎ@定義}

\ifTombow\pagebreak\fi
\begin{lstlisting}[style=grammar]
// 関数の宣言
void f( ) ;

// 宣言
void f( )
// 定義
{ }
\end{lstlisting}

関数の宣言というのは何度書いても大丈夫だ。

\begin{lstlisting}[style=grammar]
// 宣言
int f( int x ) ;

// 再宣言
int f( int x ) ;

// 再宣言
int f( int x ) ;
\end{lstlisting}

関数の宣言というのは戻り値の型や関数名や引数リストだけで、\texttt{";"}\,で終わる。

関数の定義とは、関数の宣言のあとの\,\texttt{"\{\}"}\,だ。この場合、宣言のあとに\,\texttt{";"}\,は書かない。

\begin{lstlisting}[style=grammar]
int f( int x ) { return x ; }
\end{lstlisting}

関数の定義は一度しか書けない。

\begin{lstlisting}[style=grammar]
// 定義
void f() {}
// エラー、 再定義
void f() {}
\end{lstlisting}

なぜ関数は宣言と定義とに分かれているかというと、C++では名前は宣言しないと使えないためだ。

\begin{lstlisting}[language={C++}]
int main()
{
    // エラー
    // 名前fは宣言されていない
    f() ;
}

// 定義
void f() { }
\end{lstlisting}

なので、必ず名前は使う前に宣言しなければならない。

\begin{lstlisting}[language={C++}]
// 名前fの宣言
void f() ;

int main()
{
    // OK、名前fは関数
    f() ;
}

// 名前fの定義
void f() { }
\end{lstlisting}

さて、話を元に戻そう。これから学ぶのは\(n\)回\,\texttt{"hello{\textbackslash}n"s}\,と出力するプログラムの書き方だ。ただし\(n\)はユーザーが入力するので実行時にしかわからない。すでに我々はユーザーから\(n\)の入力を受け取る部分のプログラムは書いた。

\begin{lstlisting}[language={C++}]
// n回出力する関数の宣言
void hello_n( int n ) ;

int main()
{
    // ユーザーからの入力
    int n {} ;
    std::cin >> n ;
    // n回出力
    hello_n( n ) ;
}
\end{lstlisting}

あとは関数\texttt{hello\_n(n)}が\(n\)回\,\texttt{"hello{\textbackslash}n"s}と出力するようなループを実行すればいいのだ。

すでに我々は無限回\,\texttt{"hello{\textbackslash}n"s}と出力する方法を知っている。まずは無限回ループを書こう。

\begin{lstlisting}[language={C++}]
void hello_n( int n )
{
loop :
    std::cout << "hello\n"s ;
    goto loop ;
}
\end{lstlisting}

終了条件付きループで学んだように、このループを\(n\)回繰り返した場合に終了させるには、\texttt{if文}を使って、終了条件に達したかどうかで実行を分岐させればよい。

\ifTombow\pagebreak\fi
\begin{lstlisting}[language={C++}]
void hello_n( int n )
{
loop :
    // まだn回繰り返していない場合
    if ( ??? )
    { // 以下を実行
        std::cout << "hello\n"s ;
        goto loop ;
    }
}
\end{lstlisting}

このコードを完成させるにはどうすればいいのか。まず、現在何回繰り返しを行ったのか記録する必要がある。このために変数を作る。

\begin{lstlisting}[language={C++}]
int i = 0 ;
\end{lstlisting}

変数\texttt{i}の初期値は0だ。まだ繰り返し実行を1回も行っていないということは、つまり0回繰り返し実行をしたということだ。

1回繰り返し実行をするたびに、変数\texttt{i}の値を1増やす。

\begin{lstlisting}[language={C++}]
i = i + 1 ;
\end{lstlisting}

これはすでに学んだように、もっと簡単に書ける。

\begin{lstlisting}[language={C++}]
i += 1 ;
\end{lstlisting}

実は、さらに簡単に書くこともできる。変数の代入前の値に1を足した値を代入する、つまり変数の値を1増やすというのはとてもよく書くコードなので、とても簡単な演算子が用意されている。\texttt{operator ++}だ。
\index{\protect{++}@\texttt{\protect{++}}}

\begin{lstlisting}[language={C++}]
int main()
{
    int i = 0 ;
    ++i ; // 1
    ++i ; // 2
    ++i ; // 3
}
\end{lstlisting}

これで変数\texttt{i}の値は1増える。これをインクリメント(increment)\index{いんくりめんと@インクリメント}という。

インクリメントと対になるのがデクリメント(decrement)\index{でくりめんと@デクリメント}だ。これは変数の値を1減らす。演算子は\texttt{operator {-}{-}}\,だ。
\index{{-}{-}@\texttt{{-}{-}}}

\ifTombow\pagebreak\fi
\begin{lstlisting}[language={C++}]
int main()
{
    int i = 0 ;
    --i ; // -1
    --i ; // -2
    --i ; // -3
}
\end{lstlisting}

さて、必要な知識は学び終えたので本題に戻ろう。\(n\)回の繰り返しをしたあとにループを終了するには、まずいま何回繰り返し実行しているのかを記録する必要がある。その方法を学ぶために、0, 1, 2, 3, 4\ldots と無限に出力されるプログラムを書いてみよう。

このプログラムを実行すると以下のように表示される。

\begin{lstlisting}[style=terminal]
$ make run
1, 2, 3, 4, 5, 6, [Ctrl-C]
\end{lstlisting}

\texttt{Ctrl-C}を押すまでプログラムは無限に実行される。

ではどうやって書くのか。以下のようにする。

\begin{enumerate}
\def\labelenumi{\arabic{enumi}.}
\item
  変数\texttt{i}を作り、値を\texttt{0}にする
\item
  変数\texttt{i}と\,\texttt{", "s}を出力する
\item
  変数\texttt{i}をインクリメントする
\item
  \texttt{goto} 2.
\end{enumerate}

この処理を素直に書くと以下のコードになる。

\begin{lstlisting}[language={C++}]
int main()
{
    // 1. 変数iを作り、値を0にする
    int i = 0 ;
loop :
    // 2. 変数iと", "sを出力する
    std::cout << i << ", "s ;
    // 3. 変数iをインクリメントする
    ++i ;
    // 4. goto 2
    goto loop ;
}
\end{lstlisting}

どうやら、いま何回繰り返し実行しているか記録することはできるようになったようだ。

ここまでくればしめたもの。あとは\texttt{goto文}\index{goto@\texttt{goto}文}を実行するかどうかを\texttt{if文}\index{if@\texttt{if}文}で条件分岐すればよい。しかし、\texttt{if文}の中にどんな条件を書けばいいのだろうか。

\ifTombow\pagebreak\fi
\begin{lstlisting}[language={C++}]
void hello_n( int n )
{
    int i = 0 ;
loop :
    // まだn回繰り返し実行をしていなければ実行
    if ( ??? )
    {
        std::cout << "hello\n"s ;
        ++i ;
        goto loop ;
    }
}
\end{lstlisting}

具体的に考えてみよう。\texttt{n == 3}のとき、つまり3回繰り返すときを考えよう。

\begin{enumerate}
\def\labelenumi{\arabic{enumi}.}
\item
  1回目の\texttt{if}文実行のとき、\texttt{i == 0}
\item
  2回目の\texttt{if}文実行のとき、\texttt{i == 1}
\item
  3回目の\texttt{if}文実行のとき、\texttt{i == 2}
\item
  4回目の\texttt{if}文実行のとき、\texttt{i == 3}
\end{enumerate}

ここでは\texttt{n == 3}なので、3回まで実行してほしい。つまり3回目までは\texttt{true}になり、4回目の\texttt{if}文実行のときには\texttt{false}になるような式を書く。そのような式とは、ズバリ\,\texttt{"i != n"}\,だ。

\begin{lstlisting}[language={C++}]
void hello_n( int n )
{
    int i = 0 ;
loop :
    if ( i != n )
    {
        std::cout << "hello\n"s ;
        ++i ;
        goto loop ;
    }
}
\end{lstlisting}

さっそく実行してみよう。

\begin{lstlisting}[style=terminal]
$ make run
3
hello
hello
hello
(@\ifTombow\pagebreak\fi@)
$ make run
2
hello
hello
\end{lstlisting}

なるほど、動くようだ。しかしこのプログラムにはバグがある。\texttt{-1}を入力すると、なぜか大量の\texttt{hello}が出力されてしまうのだ。

\begin{lstlisting}[style=terminal]
$ make run
-1
hello
hello
hello
hello
[Ctrl-C]
\end{lstlisting}

この原因はまだ現時点の読者には難しい。この謎はいずれ明らかにするとして、いまは\texttt{n}が負数の場合にプログラムを0回の繰り返し分の実行で終了するように書き換えよう。

\begin{lstlisting}[language={C++}]
void hello_n( int n )
{
    // nが負数ならば
    if ( n < 0 )
        // 関数の実行を終了
        return ;

    int i = 0 ;
loop :
    if ( i != n )
    {
        std::cout << "hello\n"s ;
        ++i ;
        goto loop ;
    }
}
\end{lstlisting}

\ifTombow\pagebreak\fi
\hypersection{ch0803}{while文}
\index{while@\texttt{while}文}\index{るぷ@ループ!while@\texttt{while}文}

\texttt{goto文}は極めて原始的で使いづらい機能だ。現実のC++プログラムでは\texttt{goto文}はめったに使われない。もっと簡単な機能を使う。ではなぜ\texttt{goto文}が存在するかというと、\texttt{goto文}は最も原始的で基礎的で、ほかの繰り返し機能は\texttt{if文}と\texttt{goto文}に変換することで実現できるからだ。

\texttt{goto文}より簡単な繰り返し文に、\texttt{while文}がある。ここでは\texttt{goto文}と\texttt{while文}を比較することで、\texttt{while文}を学んでいこう。

\hypersubsection{ch080301}{無限ループ}
\index{むげんるぷ@無限ループ}\index{るぷ@ループ!むげん@無限〜}

無限ループを\texttt{goto文}で書く方法を思い出してみよう。

\begin{lstlisting}[language={C++}]
int main()
{
    auto hello = []()
    { std::cout << "hello\n"s ; } ;

loop :
    // 繰り返し実行される文
    hello() ;
    goto loop ;
}
\end{lstlisting}

このコードで本当に重要なのは関数\texttt{hello}を呼び出している部分だ。ここが繰り返し実行される文で、\texttt{ラベル文}と\texttt{goto文}は、繰り返し実行を実現するために必要な記述でしかない。

そこで\texttt{while(true)}\index{while(true)@\texttt{while(true)}}だ。\texttt{while(true)}は\texttt{goto文}と\texttt{ラベル文}よりも簡単に無限ループを実現できる。

\begin{lstlisting}[style=grammar]
while (true) 文
\end{lstlisting}

\texttt{while文}は文を無限に繰り返して実行してくれる。試してみよう。

\begin{lstlisting}[language={C++}]
int main()
{
    auto hello = []()
    { std::cout << "hello\n"s ; } ;

    while (true)
        hello() ;
}
\end{lstlisting}

このコードの重要な部分は以下の2行。

\begin{lstlisting}[language={C++}]
while (true)
    hello() ;
\end{lstlisting}

これを\texttt{goto文}と\texttt{ラベル文}を使った無限ループと比べてみよう。

\begin{lstlisting}[language={C++}]
loop:
    hello() ;
    goto loop ;
\end{lstlisting}

どちらも同じ意味のコードだが、\texttt{while文}の方が明らかに書きやすくなっているのがわかる。

\texttt{goto文}で学んだ、ユーザーからの整数値の入力の合計の計算を繰り返すプログラムを\texttt{while(true)}で書いてみよう。

\begin{lstlisting}[language={C++}]
int input()
{
    std::cout << ">"s ;
    int x {} ;
    std::cin >> x ;
    return x ;
}

int main()
{
    int sum = 0 ;

    while( true )
    {
        sum += input() ;
        std::cout << sum << "\n"s ;
    }
}
\end{lstlisting}

重要なのは以下の5行だ。

\begin{lstlisting}[language={C++}]
while( true )
{
    sum += input() ;
    std::cout << sum << "\n"s ;
}
\end{lstlisting}

これを\texttt{goto文}で書いた場合と比べてみよう。

\ifTombow\pagebreak\fi
\begin{lstlisting}[language={C++}]
loop :
    sum += input() ;
    std::cout << sum << "\n"s ;
    goto loop ;
\end{lstlisting}

本当に重要で本質的な、繰り返し実行をする部分の2行のコードはまったく変わっていない。それでいて\texttt{while(true)}の方が圧倒的に簡単に書ける。

\hypersubsection{ch080302}{終了条件付きループ}
\index{しゆうりようじようけんつきるぷ@終了条件付きループ}\index{るぷ@ループ!しゆうりようじようけんつき@終了条件付き〜}

なるほど、無限ループを書くのに、\texttt{goto文}を使うより\texttt{while(true)}を使った方がいいことがわかった。ではほかのループの場合でも、\texttt{while文}の方が使いやすいだろうか。

本書を先頭から読んでいる優秀な読者は\texttt{while(true)}の\texttt{true}は\texttt{bool}型の値であることに気が付いているだろう。実は\texttt{while(E)}の括弧の中\texttt{E}は、\texttt{if(E)}と書くのとまったく同じ\texttt{条件}なのだ。\texttt{条件}が\texttt{true}であれば繰り返し実行される。\texttt{false}なら繰り返し実行されない。

\begin{lstlisting}[style=grammar]
while ( 条件 ) 文
\end{lstlisting}

\begin{lstlisting}[language={C++}]
int main()
{
    // 実行されない
    while ( false )
        std::cout << "No"s ;

    // 実行されない
    while ( 1 > 2 )
        std::cout << "No"s ;

    // 実行される
    // 無限ループ
    while ( 1 < 2 )
        std::cout << "Yes"s ;
}
\end{lstlisting}

\texttt{while文}を使って、\texttt{0}が入力されたら終了する合計値計算プログラムを書いてみよう。

\begin{lstlisting}[language={C++}]
int input()
{
    std::cout << ">"s ;
    int x {} ;
    std::cin >> x ;
    return x ;
}

int main()
{
    int sum = 0 ;
    int x {} ;

    while( ( x = input() ) != 0 )
    {
        sum += x ;
        std::cout << sum << "\n"s ;
    }
}
\end{lstlisting}

重要なのはこの5行。

\begin{lstlisting}[language={C++}]
while( ( x = input() ) != 0 )
{
    sum += x ;
    std::cout << sum << "\n"s ;
}
\end{lstlisting}

ここではちょっと難しいコードが出てくる。\texttt{while}の中の\texttt{条件}が、\texttt{"( x = input() ) != 0"}\,になっている。これはどういうことか。

実は\texttt{条件}は\texttt{bool型}に変換さえできればどんな式でも書ける。

\begin{lstlisting}[language={C++}]
int main()
{
    int x { } ;

    if ( (x = 1) == 1 )
        std::cout << "(x = 1) is 1.\n"s ;
}
\end{lstlisting}

このコードでは、``\texttt{(x=1)}''と``\texttt{1}''が等しい``\texttt{==}''かどうかを判断している。``\texttt{(x=1)}''という式は変数\texttt{x}に\texttt{1}を代入する式だ。\texttt{代入式}の値は、代入された変数の値になる。この場合変数\texttt{x}の値だ。変数\texttt{x}には\texttt{1}が代入されているので、その値は1、つまり``\texttt{(x=1) == 1}''は``\texttt{1 == 1}''と書くのと同じ意味になる。この結果は\texttt{true}だ。

さて、このことを踏まえて、``\texttt{( x = input() ) != 0}''を考えてみよう。

``\texttt{( x = input() )}''は変数\texttt{x}に関数\texttt{input}を呼び出した結果を代入している。関数\texttt{input}はユーザーから入力を得て、その入力をそのまま返す。つまり変数\texttt{x}にはユーザーの入力した値が代入される。その結果が\texttt{0}と等しくない``\texttt{!=}''かどうかを判断している。つまり、ユーザーが\texttt{0}を入力した場合は\texttt{false}、非ゼロを入力した場合は\texttt{true}となる。

\texttt{while(条件)}は\texttt{条件}が\texttt{true}となる場合に繰り返し実行をする。結果として、ユーザーが\texttt{0}を入力するまで繰り返し実行をするコードになる。

\texttt{goto文}を使った終了条件付きループと比較してみよう。

\begin{lstlisting}[language={C++}]
loop:
    if ( (x = input() ) != 0 )
    {
        sum += x ;
        std::cout << sum << "\n"s ;
        goto loop ;
    }
\end{lstlisting}

\texttt{while文}の方が圧倒的に書きやすいことがわかる。

\hypersubsection{ch080303}{インデックスループ}
\index{いんでつくするぷ@インデックスループ}\index{るぷ@ループ!いんでつくす@インデックス〜}

\(n\)回\,\texttt{"hello{\textbackslash}n"s}と出力するプログラムを\texttt{while文}で書いてみよう。ただし\(n\)はユーザーが入力するものとする。

まずは\texttt{goto文}でも使ったループ以外の処理をするコードから。

\begin{lstlisting}[language={C++}]
void hello_n( int n ) ;

int main()
{
    int n {} ;
    std::cin >> n ;
    hello_n( n ) ;
}
\end{lstlisting}

あとは関数\texttt{hello\_n(n)}がインデックスループを実装するだけだ。ただし\texttt{n}が負数ならば何も実行しないようにしよう。

\texttt{goto文}でインデックスループを書くときに学んだように、
\begin{enumerate}
\def\labelenumi{\arabic{enumi}.}
\item
  \texttt{n < 0}ならば関数を終了
\item
  変数\texttt{i}を作り値を0にする
\item
  \texttt{i != n}ならば繰り返し実行
\item
  出力
\item
  \texttt{++i}
\item
  \texttt{goto} 3.
\end{enumerate}
を\texttt{while文}で書くだけだ。

\ifTombow\pagebreak\fi
\begin{lstlisting}[language={C++}]
void hello_n( int n )
{
    // 1. n < 0ならば関数を終了
    if ( n < 0 )
        return ;

    // 2. 変数iを作り値を0にする
    int i = 0 ;

    // 3. i != nならば繰り返し実行
    while( i != n )
    {   // 4. 出力
        std::cout << "hello\n"s ;
        // 5. ++i
        ++i ;
    } // 6. goto 3
}
\end{lstlisting}

重要な部分だけ抜き出すと以下のとおり。

\begin{lstlisting}[language={C++}]
while( i != n )
{
    std::cout << "hello\n"s ;
    ++i ;
}
\end{lstlisting}

\texttt{goto文}を使ったインデックスループと比較してみよう。

\begin{lstlisting}[language={C++}]
loop :
    if ( i != n )
    {
        std::cout << "hello\n"s ;
        ++i ;
        goto loop ;
    }
\end{lstlisting}

読者の中にはあまり変わらないのではないかと思う人もいるかもしれない。しかし、次の問題を解くプログラムを書くと、\texttt{while文}がいかに楽に書けるかを実感するだろう。

\ifTombow\pagebreak\fi
\textsf{問題}:以下のような九九の表を出力するプログラムを書きなさい。

\begin{lstlisting}[style=terminal]
1   2   3   4   5   6   7   8   9   
2   4   6   8   10  12  14  16  18  
3   6   9   12  15  18  21  24  27  
4   8   12  16  20  24  28  32  36  
5   10  15  20  25  30  35  40  45  
6   12  18  24  30  36  42  48  54  
7   14  21  28  35  42  49  56  63  
8   16  24  32  40  48  56  64  72  
9   18  27  36  45  54  63  72  81
\end{lstlisting}

もちろん、このような文字列を愚直に出力しろという問題ではない。

\begin{lstlisting}[language={C++}]
int main()
{
    // 違う!
    std::cout << "1 2 3 4 5..."s ;
}
\end{lstlisting}

逐次実行、条件分岐、ループまでを習得した誇りある本物のプログラマーである我々は、もちろん九九の表はループを書いて出力する。

まず出力すべき表を見ると、数値が左揃えになっていることに気が付くだろう。

\begin{lstlisting}[style=terminal]
4   8   12
5   10  15
\end{lstlisting}

\texttt{8}は1文字、\texttt{10}は2文字にもかかわらず、\texttt{12}と\texttt{15}は同じ列目から始まっている。これは出力するスペース文字を調整することでも実現できるが、ここでは単にタブ文字を使っている。

タブ文字は\texttt{Makefile}を書くのにも使った文字で、C++の文字列中に直接書くこともできるが、エスケープ文字\,\texttt{{\textbackslash}t}を使ってもよい。

\begin{lstlisting}[language={C++}]
int main()
{
    std::cout << "4\t8\t12\n5\t10\t15"s ;
}
\end{lstlisting}

エスケープ文字\,\texttt{{\textbackslash}n}が改行文字に置き換わるように、エスケープ文字\,\texttt{{\textbackslash}t}\index{{\textbackslash}t@\texttt{{\textbackslash}t}}はタブ文字\index{たぶもじ@タブ文字}に置き換わる。

九九の表はどうやって出力すればよいだろうか。計算自体はC++では\,\texttt{"a*b"}\,でできる。上の表がどのように計算されているかを考えてみよう。

\ifTombow\pagebreak\fi
\begin{lstlisting}[style=terminal]
1*1 1*2 1*3 1*4 1*5 1*6 1*7 1*8 1*9 
2*1 2*2 2*3 2*4 2*5 2*6 2*7 2*8 2*9 
3*1 3*2 3*3 3*4 3*5 3*6 3*7 3*8 3*9 
4*1 4*2 4*3 4*4 4*5 4*6 4*7 4*8 4*9 
5*1 5*2 5*3 5*4 5*5 5*6 5*7 5*8 5*9 
6*1 6*2 6*3 6*4 6*5 6*6 6*7 6*8 6*9 
7*1 7*2 7*3 7*4 7*5 7*6 7*7 7*8 7*9 
8*1 8*2 8*3 8*4 8*5 8*6 8*7 8*8 8*9 
9*1 9*2 9*3 9*4 9*5 9*6 9*7 9*8 9*9
\end{lstlisting}

これを見ると、\texttt{"a*b"}\,のうちの\texttt{a}を\texttt{1}から\texttt{9}までインクリメントし、それに対して\texttt{b}を\texttt{1}から\texttt{9}までインクリメントさせればよい。つまり、9回のインデックスループの中で9回のインデックスループを実行することになる。ループの中のループだ。

\begin{lstlisting}[style=grammar]
while ( 条件 )
    while ( 条件 )
        文
\end{lstlisting}

さっそくそのようなコードを書いてみよう。

\begin{lstlisting}[language={C++}]
int main()
{
    // 1から9まで
    int a = 1 ;
    while ( a <= 9 )
    {
        // 1から9まで
        int b = 1 ;
        while ( b <= 9 )
        {
            // 計算結果を出力
            std::cout << a * b << "\t"s ;
            ++b ;
        }
        // 段の終わりに改行
        std::cout << "\n"s ;
        ++a ;
    }
}
\end{lstlisting}

うまくいった。

ところで、このコードを\texttt{goto文}で書くとどうなるだろうか。

\ifTombow\pagebreak\fi
\begin{lstlisting}[language={C++}]
int main()
{
    int a = 1 ;
loop_outer :
    if ( a <= 9 )
    {
        int b = 1 ;
loop_inner :
        if ( b <= 9 )
        {
            std::cout << a * b << "\t"s ;
            ++b ;
            goto loop_inner ;
        }
        std::cout << "\n"s ;
        ++a ;
        goto loop_outer ;
    }
}
\end{lstlisting}

とてつもなく読みにくい。

\hypersection{ch0804}{for文}
\index{for@\texttt{for}文}\index{るぷ@ループ!for@\texttt{for}文}

ところでいままで\texttt{while文}で書いてきたインデックスループには特徴がある。

試しに1から100までの整数を出力するコードを見てみよう。

\begin{lstlisting}[language={C++}]
int main()
{
    int i = 1 ;
    while ( i <= 100 )
    {
        std::cout << i << " "s ;
        ++i ;
    }
}
\end{lstlisting}

このコードを読むと、以下のようなパターンがあることがわかる。

\begin{lstlisting}[language={C++}]
int main()
{
    // ループ実行前の変数の宣言と初期化
    int i = 1 ;
(@\ifTombow\pagebreak\fi@)
    // ループ中の終了条件の確認
    while ( i <= 100 )
    {
        // 実際に繰り返したい文
        std::cout << i << " "s ;
        // 各ループの最後に必ず行う処理
        ++i ;
    }
}
\end{lstlisting}

ここで真に必要なのは、「実際に繰り返したい文」だ。その他の処理は、ループを実現するために必要なコードだ。ループの実現に必要な処理が飛び飛びの場所にあるのは、はなはだわかりにくい。

\texttt{for文}はそのような問題を解決するための機能だ。

\begin{lstlisting}[style=grammar]
for ( 変数の宣言 ; 終了条件の確認 ; 各ループの最後に必ず行う処理 ) 文
\end{lstlisting}

\texttt{for文}を使うと、上のコードは以下のように書ける。

\begin{lstlisting}[language={C++}]
int main()
{
    for ( int i = 1 ; i <= 100 ; ++i )
    {
        std::cout << i << " "s ;
    } 
}
\end{lstlisting}

ループの実現に必要な部分だけ抜き出すと以下のようになる。

\begin{lstlisting}[language={C++}]
// for文の開始
for (
// 変数の宣言と初期化
int i = 1 ;
// 終了条件の確認
i <= 100 ;
// 各ループの最後に必ず行う処理
++i )
\end{lstlisting}

\texttt{for文}はインデックスループによくあるパターンをわかりやすく書くための機能だ。例えば\texttt{while文}のときに書いた九九の表を出力するプログラムは、\texttt{for文}ならばこんなに簡潔に書ける。

\ifTombow\pagebreak\fi
\begin{lstlisting}[language={C++}]
int main()
{
    for ( int a = 1 ; a <= 9 ; ++a )
    {
        for ( int b = 1 ; b <= 9 ; ++b )
        { std::cout << a*b << "\t"s ; }

        std::cout << "\n"s ;
    }
}
\end{lstlisting}

\texttt{while文}を使ったコードと比べてみよう。

\begin{lstlisting}[language={C++}]
int main()
{
    int a = 1 ;
    while ( a <= 9 )
    {
        int b = 1 ;
        while ( b <= 9 )
        {
            std::cout << a * b << "\t"s ;
            ++b ;
        }
        std::cout << "\n"s ;
        ++a ;
    }
}
\end{lstlisting}

格段に読みやすくなっていることがわかる。

C++ではカンマ\,\texttt{','}\,\index{,@\texttt{,}}を使うことで、複数の\texttt{式}を1つの\texttt{文}に書くことができる。

\begin{lstlisting}[language={C++}]
int main()
{
    int a = 0, b = 0 ;
    ++a, ++b ;
}
\end{lstlisting}

\texttt{for文}でもカンマが使える。九九の表を出力するプログラムは、以下のように書くこともできる。

\ifTombow\pagebreak\fi
\begin{lstlisting}[language={C++}]
int main()
{
    for ( int a = 1 ; a <= 9 ; ++a, std::cout << "\n"s )
        for ( int b = 1 ; b <= 9 ; ++b )
            std::cout << a*b << "\t"s ;
}
\end{lstlisting}

変数もカンマで複数宣言できると知った読者は、以下のように書きたくなるだろう。

\begin{lstlisting}[language={C++}]
int main()
{
    for (   int a = 1, b = 1 ;
            a <= 9 ;
            ++a, ++b,
            std::cout << "\n"s
        )
            std::cout << a*b << "\t"s ;
}
\end{lstlisting}

これは動かない。なぜならば、\texttt{for文}を2つネストさせたループは、\(a \times b\)回のループで、変数\texttt{a}が\texttt{1}から\texttt{9}まで変化するそれぞれに対して、変数\texttt{b}が\texttt{1}から\texttt{9}まで変化する。しかし、上の\texttt{for文}1つのコードは、変数\texttt{a}, \texttt{b}ともに同時に\texttt{1}から\texttt{9}まで変化する。したがって、これは単に\texttt{a}回のループでしかない。\texttt{a}回のループの中で\texttt{b}回のループをすることで\(a \times b\)回のループを実現できる。

\texttt{for文}では使わない部分を省略することができる。

\begin{lstlisting}[language={C++}]
int main()
{
    bool b = true ;
    // for文による変数宣言は使わない
    for ( ; b ; b = false )
        std::cout << "hello"s ;
}
\end{lstlisting}

\texttt{for文}で終了条件を省略した場合、\texttt{true}と同じになる。
\index{for@\texttt{for}文!しようりやく@省略}

\begin{lstlisting}[language={C++}]
int main()
{
    for (;;)
        std::cout << "hello\n"s ;
}
\end{lstlisting}

このプログラムは\,\texttt{"hello{\textbackslash}n"s}と無限に出力し続けるプログラムだ。\texttt{"for(;;)"}\,は\,\texttt{"for(;true;)"}\,と同じ意味であり、\texttt{"while(true)"}\,とも同じ意味だ。

\hypersection{ch0805}{do文}
\index{do@\texttt{do}文}\index{るぷ@ループ!do@\texttt{do}文}

\texttt{do文}は\texttt{while文}に似ている。

\begin{lstlisting}[style=grammar]
do 文 while ( 条件 ) ;
\end{lstlisting}

比較のために\texttt{while文}の文法も書いてみると以下のようになる。

\begin{lstlisting}[style=grammar]
while ( 条件 ) 文
\end{lstlisting}

\texttt{while文}はまず\texttt{条件}を確認し\texttt{true}の場合\texttt{文}を実行する。これを繰り返す。

\begin{lstlisting}[language={C++}]
int main()
{
    while ( false )
    {
        std::cout << "hello\n"s ;
    }
}
\end{lstlisting}

\texttt{do文}はまず\texttt{文}を実行する。しかる後に\texttt{条件}を確認し\texttt{true}の場合繰り返しを行う。

\begin{lstlisting}[language={C++}]
int main()
{
    do {
        std::cout << "hello\n"s ;
    } while ( false ) ;
}
\end{lstlisting}

違いがわかっただろうか。\texttt{do文}は繰り返し実行する\texttt{文}を、\texttt{条件}がなんであれ、最初に一度実行する。

\texttt{do文}を使うと条件にかかわらず文を1回は実行するコードが、文の重複なく書けるようになる。

\ifTombow\pagebreak\fi
\hypersection{ch0806}{break文}
\index{break@\texttt{break}文}\index{るぷ@ループ!break@\texttt{break}文}

ループの実行の途中で、ループの中から外に脱出\index{るぷ@ループ!だつしゆつ@脱出}したくなった場合、どうすればいいのだろうか。例えばループを実行中に何らかのエラーを検出したので処理を中止したい場合などだ。

\begin{lstlisting}[language={C++}]
while ( true )
{
    // 処理

    if ( is_error() )
        // エラーのため脱出したくなった

    // 処理
}
\end{lstlisting}

\texttt{break文}はループの途中から脱出するための文だ。

\begin{lstlisting}[style=grammar]
break ;
\end{lstlisting}

\texttt{break文}は\texttt{for文}、\texttt{while文}、\texttt{do文}の中でしか使えない。

\texttt{break文}は\texttt{for文}、\texttt{while文}、\texttt{do文}の外側に脱出する。

\begin{lstlisting}[language={C++}]
int main()
{
    while ( true )
    {
        // 処理

        break ;

        // 処理
    }
}
\end{lstlisting}

これは以下のようなコードと同じだ。

\begin{lstlisting}[language={C++}]
int main()
{
    while ( true )
    {
        // 処理

        goto break_while ;

        // 処理
    }
break_while : ;
}
\end{lstlisting}

\texttt{break文}は最も内側の\texttt{繰り返し文}から脱出する

\begin{lstlisting}[language={C++}]
int main()
{
    while ( true ) // 外側
    {
        while ( true ) // 内側
        {
            break ;
        }
        // ここに脱出
    }
}
\end{lstlisting}

\hypersection{ch0807}{continue文}
\index{continue@\texttt{continue}文}\index{るぷ@ループ!continue@\texttt{continue}文}

ループの途中で、いまのループを打ち切って\index{るぷ@ループ!うちきり@打ち切り}次のループに進みたい場合はどうすればいいのだろう。例えば、ループの途中でエラーを検出したので、そのループについては処理を打ち切りたい場合だ。

\begin{lstlisting}[language={C++}]
while ( true )
{
    // 処理

    if ( is_error() )
        // このループは打ち切りたい

    // 処理
}
\end{lstlisting}

\texttt{continue文}はループを打ち切って次のループに行くための文だ。

\begin{lstlisting}[style=grammar]
continue ;
\end{lstlisting}

\texttt{continue文}は\texttt{for文}、\texttt{while文}、\texttt{do}文の中でしか使えない。

\ifTombow\pagebreak\fi
\begin{lstlisting}[language={C++}]
int main()
{
    while ( true )
    {
        // 処理

        continue ;

        // 処理
    }
}
\end{lstlisting}

これは以下のようなコードと同じだ。

\begin{lstlisting}[language={C++}]
int main()
{
    while ( true )
    {
        // 処理

        goto continue_while ;

        // 処理

continue_while : ;
    }
}
\end{lstlisting}

\texttt{continue}文はループの最後に処理を移す。その結果、次のループを実行するかどうかの\texttt{条件}を評価することになる。

\texttt{continue文}は最も内側のループに対応する。

\begin{lstlisting}[language={C++}]
int main()
{
    while ( true ) // 外側
    {
        while ( true ) // 内側
        {
            continue ;
            // continueはここに実行を移す
        }
    }
}
\end{lstlisting}

\hypersection{ch0808}{再帰関数}
\index{さいきかんすう@再帰関数}\index{るぷ@ループ!さいきかんすう@再帰関数}

最後に関数でループを実装する方法を示してこの章を終わりにしよう。

関数は関数を呼び出すことができる。

\begin{lstlisting}[language={C++}]
void f() { }

void g()
{
    f() ; // 関数fの呼び出し
}

int main()
{
    g() ; // 関数gの呼び出し
}
\end{lstlisting}

ではもし、関数が自分自身を呼び出したらどうなるだろうか。

\begin{lstlisting}[language={C++}]
void hello()
{
    std::cout << "hello\n" ;
    hello() ;
}

int main()
{
    hello() ;
}
\end{lstlisting}

\begin{enumerate}
\def\labelenumi{\arabic{enumi}.}
\item
  関数\texttt{main}は関数\texttt{hello}を呼び出す
\item
  関数\texttt{hello}は\,\texttt{"hello{\textbackslash}n"}\,と出力して関数\texttt{hello}を呼び出す
\end{enumerate}

関数\texttt{hello}は必ず関数\texttt{hello}を呼び出すので、この実行は無限ループする。

関数が自分自身を呼び出すことを、\texttt{再帰}(recursion)\index{さいき@再帰}という。

なるほど、再帰によって無限ループを実現できることはわかった。では終了条件付きループは書けるだろうか。

関数は\texttt{return文}\index{return@\texttt{return}文}によって呼び出し元に戻る。単に\,\texttt{'return ;'}\,と書けば再帰はしない。そして、\texttt{if文}によって実行は分岐できる。これを使えば再帰で終了条件付きループが実現できる。

試しに、ユーザーが\texttt{0}を入力するまでループし続けるプログラムを書いてみよう。

\ifTombow\pagebreak\fi
\begin{lstlisting}[language={C++}]
// ユーザーからの入力を返す
int input ()
{
    int x { } ;
    std::cin >> x ;
    return x ;
}

// 0の入力を終了条件としたループ
void loop_until_zero()
{
    if ( input() == 0 )
        return ;
    else
        loop_until_zero() ;
}

int main()
{
    loop_until_zero() ;
}
\end{lstlisting}

書けた。

ではインデックスループはどうだろうか。\texttt{1}から\texttt{10}までの整数を出力してみよう。

インデックスループを実現するには、書き換えられる変数が必要だ。関数は引数で値を渡すことができる。

\begin{lstlisting}[language={C++}]
void g( int x ) { }
void f( int x ) { g( x+1 ) ; }

int main() { f( 0 ) ; }
\end{lstlisting}

これを見ると、関数\texttt{main}は関数\texttt{f}に引数\texttt{0}を渡し、関数\texttt{f}は関数\texttt{g}に引数\texttt{1}を渡している。これをもっと再帰的に考えよう。

\begin{lstlisting}[language={C++}]
void until_ten( int x )
{
    if ( x > 10 )
        return ;
    else
    {
        std::cout << x << "\n" ;
        return until_ten( x + 1 ) ;
    }
}

int main()
{
    until_ten(1) ;
}
\end{lstlisting}

関数\texttt{main}は関数\texttt{until\_ten}に引数1を渡す。

関数\texttt{until\_ten}は引数が\texttt{10}より大きければ何もせず処理を戻し、そうでなければ引数を出力して再帰する。そのとき引数は\(+1\)される。

これによりインデックスループが実現できる。

関数は戻り値を返すことができる。再帰で戻り値を使うことにより面白い問題も解くことができる。

例えば、\texttt{1}と\texttt{0}だけを使った10進数\index{10しんすう@10進数}の整数を2進数\index{2しんすう@2進数}に変換するプログラムを書いてみよう。

\begin{lstlisting}[style=terminal]
$ make run
> 0
0
> 1
1
> 10
2
> 11
3
> 1010
10
> 1111
15
\end{lstlisting}

まず10進数と2進数を確認しよう。数学的に言うと「10を底にする」とか「2を底にする」という言い方をする。

具体的な例を出すと10進数では1, 2, 3, 4, 5, 6, 7, 8, 9, 0の文字を使う。\texttt{1234}は以下のようになる。
\[
1234 = 1 \times 10^3 + 2 \times 10^2 + 3 \times 10^1 + 4 \times 10^0 = 1 \times 1000 + 2 \times 100 + 3 \times 10 + 4 \times 1
\]

10進数で\texttt{1010}は以下のようになる。
\[
1010 = 1 \times 10^3 + 0 \times 10^2 + 1 \times 10^1 + 0 \times 10^0 = 1 \times 1000 + 0 \times 100 + 1 \times 10 + 0 \times 1
\]

2進数では1, 0の文字を使う。\texttt{1010}は以下のようになる。
\[
1010 = 1 \times 2^3 + 0 \times 2^2 + 1 \times 2^1 + 0 \times 2^0 = 1 \times 8 + 0 \times 4 + 1 \times 2 + 0 \times 1
\]

2進数の\texttt{1010}は10進数では\texttt{10}になる。

では問題を解いていこう。

問題を難しく考えるとかえって解けなくなる。ここではすでに10進数から2進数への変換は解決したものとして考えよう。関数\texttt{convert}によってその問題は解決した。

\begin{lstlisting}[language={C++}]
// 2進数への変換
int convert( int n ) ;
\end{lstlisting}

まだ我々は関数\texttt{convert}の中身を書いていないが、すでに書き終わったと仮定しよう。するとプログラムの残りの部分は以下のように書ける。

\begin{lstlisting}[language={C++}]
int convert( int n ) ;

// 入力
int input()
{
    std::cout << "> " ;
    int x{} ;
    std::cin >> x ;
    return x ;
}

// 出力
void output( int binary )
{
    std::cout << binary << "\n"s ;
}

int main()
{
    // 入力、 変換、 出力のループ
    while( true )
    {
        auto decimal = input() ;
        auto binary = convert( decimal ) ;
        output( binary ) ;
    } 
}
\end{lstlisting}

あとは関数\texttt{convert}を実装すればよいだけだ。

関数\texttt{convert}に引数を渡したときの結果を考えてみよう。\texttt{convert(1010)}は\texttt{10}を返し、\texttt{convert(1111)}は\texttt{15}を返す。

では\texttt{convert(-1010)}の結果はどうなるだろうか。これは\,\texttt{-10}になる。

負数と正数の違いを考えるのは面倒だ。ここでは正数を引数として与えると10進数から2進数へ変換した答えを返してくる魔法のような関数\texttt{solve}をすでに書き終えたと仮定しよう。我々はまだ関数\texttt{solve}を書いていないが、その問題は未来の自分に押し付けよう。

\ifTombow\pagebreak\fi
\begin{lstlisting}[language={C++}]
// 1,0のみを使った10進数から
// 2進数へ変換する関数
int solve( int n ) ;
\end{lstlisting}

すると、関数\texttt{convert}がやるのは負数と正数の処理だけでよい。

\begin{enumerate}
\def\labelenumi{\arabic{enumi}.}
\item
  引数が正数の場合はそのまま関数\texttt{solve}に渡して\texttt{return}
\item
  引数が負数の場合は絶対値を関数\texttt{solve}に渡して負数にして\texttt{return}
\end{enumerate}

\begin{lstlisting}[language={C++}]
int convert( int n )
{
    // 引数が正数の場合
    if ( n > 0 )
        // そのまま関数solveに渡してreturn
        return solve( n ) ;
    else // 引数が負数の場合
        // 絶対値を関数solveに渡して負数にしてreturn
        return - solve( -n ) ;
}
\end{lstlisting}

\texttt{n}が負数の場合の絶対値は\,\texttt{-n}で得られる。その場合、関数\texttt{solve}の答えは正数なので負数にする。

あとは関数\texttt{solve}を実装するだけだ。

今回、引数の整数を10進数で表現した場合に2, 3, 4, 5, 6, 7, 8, 9が使われている場合は考えないものとする。

\begin{lstlisting}[language={C++}]
// OK
solve(10111101) ;
// あり得ない
solve(2) ;
\end{lstlisting}

再帰で問題を解くには再帰的な考え方が必要だ。再帰的な考え方では、問題の一部のみを解き、残りは自分自身に丸投げする。

まずとても簡単な1桁の変換を考えよう。

\begin{lstlisting}[language={C++}]
solve(0) ; // 0
solve(1) ; // 1
\end{lstlisting}

引数が\texttt{0}か\texttt{1}の場合、単にその値を返すだけだ。関数\texttt{solve}には正数しか渡されないので、負数は考えなくてよい。すると、以下のように書ける。

\ifTombow\pagebreak\fi
\begin{lstlisting}[language={C++}]
int solve( int n )
{
    if ( n <= 1 )
        return n ;
    else
        // その他の場合
}
\end{lstlisting}

その他の場合とは、桁数が多い場合だ。

\begin{lstlisting}[language={C++}]
solve(10) ;  // 2
solve(11) ;  // 3
solve(110) ; // 4
solve(111) ; // 5
\end{lstlisting}

関数\texttt{solve}が解決するのは最下位桁だ。\texttt{110}の場合は\texttt{0}で、\texttt{111}の場合は\texttt{1}となる。最も右側の桁のみを扱う。数値から10進数で表記したときの最下位桁を取り出すには、10で割った余りが使える。覚えているだろうか。剰余演算子の\texttt{operator \%}を。

\begin{lstlisting}[language={C++}]
int solve( int n )
{
    if ( n <= 1 )
        return n ;
    else // 未完成
        return n%10 ;
}
\end{lstlisting}

結果は以下のようになる。

\begin{lstlisting}[language={C++}]
solve(10) ;  // 0
solve(11) ;  // 1
solve(110) ; // 0
solve(111) ; // 1
\end{lstlisting}

これで関数\texttt{solve}は最下位桁に完全に対応した。しかしそれ以外の桁はどうすればいいのだろう。

ここで再帰的な考え方が必要だ。関数\texttt{solve}はすでに最下位桁に完全に対応している。ならば次の桁を最下位桁とした数値で関数\texttt{solve}を再帰的に呼び出せばいいのではないか。

以下は\texttt{solve(n)}が再帰的に呼び出す関数だ。

\ifTombow\pagebreak\fi
\begin{lstlisting}[language={C++}]
solve(10) ;  // solve(1)
solve(11) ;  // solve(1)
solve(100) ; // solve(10)→solve(1)
solve(110) ; // solve(11)→solve(1)
solve(111) ; // solve(11)→solve(1)
\end{lstlisting}

10進数表記された数値から最下位桁を取り除いた数値にするというのは、11を1に, 111を11にする処理だ。これは数値を10で割ればよい。

\begin{lstlisting}[language={C++}]
10  / 10 ; // 1
11  / 10 ; // 1
100 / 10 ; // 10
110 / 10 ; // 11
111 / 10 ; // 11
\end{lstlisting}

10進数表記は桁が1つ上がると10倍される。だから10で割れば最下位桁が消える。ところで、我々は計算しようとしているのは2進数だ。2進数では桁が1つ上がると2倍される。なので、再帰的に関数\texttt{solve}を呼び出して得られた結果は2倍しなければならない。そして足し合わせる。

\begin{lstlisting}[language={C++}]
int solve( int n )
{
    // 1桁の場合
    if ( n <= 1 )
        return n ; // 単に返す
    else // それ以外
        return
            // 最下位桁の計算
            n%10
            // 残りの桁を丸投げする
            // 次の桁なので2倍する
            + 2 * solve( n/10 ) ;
}
\end{lstlisting}

冗長なコメントを除いて短くすると以下のとおり。

\begin{lstlisting}[language={C++}]
int solve( int n )
{
    if ( n <= 1 )
        return n ;
    else
        return n%10 + 2 * solve( n/10 ) ;
}
\end{lstlisting}

再帰ではないループで関数\texttt{solve}を実装するとどうなるのだろうか。

引数の数値が何桁あっても対応できるよう、ループで1桁ずつ処理していくのは変わらない。

もう一度2進数の計算を見てみよう。

\[
1010 = 1 \times 2^3 + 0 \times 2^2 + 1 \times 2^1 + 0 \times 2^0 = 1 \times 8 + 0 \times 4 + 1 \times 2 + 0 \times 1
\]

1桁目は0で、この値は\(0 \times 2^0\)、2桁目は1で、この値は\(1 \times 2^1\)になる。

一般に、\(i\)桁目の値は\(i桁目の数字 \times 2^{i-1}\)になる。

すると解き方としては、各桁の値を計算した和を返せばよい

\begin{lstlisting}[language={C++}]
int solve( int n )
{
    // 和
    int result = 0 ;
    // i桁目の数字に乗ずる値
    int i = 1 ;

    // 桁がなくなれば終了
    while ( n != 0 )
    {
        // 現在の桁を計算して足す
        result += n%10 * i ;
        // 次の桁に乗ずる値
        i *= 2 ;
        // 桁を1つ減らす
        n /= 10 ;
    }

    return result ;
}
\end{lstlisting}

再帰を使うコードは、再帰を理解できれば短く簡潔でわかりやすい。ただし、再帰を理解するためにはまず再帰を理解しなければならない。

再帰は万能ではない。そもそも関数とは、別の関数から呼ばれるものだ。関数\texttt{main}だけは特別で、関数\texttt{main}を呼び出すことはできない。

\begin{lstlisting}[language={C++}]
int main()
{
    main() ; // エラー
}
\end{lstlisting}

関数の実行が終了した場合、呼び出し元に処理が戻る。そのために関数は呼び出し元を覚えていなければならない。これには通常\texttt{スタック}\index{すたつく@スタック}と呼ばれるメモリーを消費する。

\ifTombow\pagebreak\fi
\begin{lstlisting}[language={C++}]
void f() { }            // gに戻る
void g() { f() ; }      // mainに戻る 
int main() { g() ; }
\end{lstlisting}

関数の中の変数も通常\texttt{スタック}に確保される。これもメモリーを消費する。

\begin{lstlisting}[language={C++}]
void f() { }

void g()
{
    int x {} ;
    std::cin >> x ;
    f() ;   // 関数を呼び出す
    // 関数を呼び出したあとに変数を使う
    std::cout << x ;
}
\end{lstlisting}

このコードでは、関数\texttt{g}が変数\texttt{x}を用意し、関数\texttt{f}を呼び出し、処理が戻ったら変数\texttt{x}を使っている。このコードが動くためには、変数\texttt{x}は関数\texttt{f}が実行されている間もスタックメモリーを消費し続けなければならない。

スタックメモリーは有限であるので、以下のような再帰による無限ループは、いつかスタックメモリーを消費し尽して実行が止まるはずだ。

\begin{lstlisting}[language={C++}]
void hello()
{
    std::cout << "hello\n" ;
    hello() ;
}

int main() { hello() ; }
\end{lstlisting}

しかし、大半の読者の環境ではプログラムの実行が止まらないはずだ。これはコンパイラーの末尾再帰の最適化によるものだ。

末尾再帰\index{さいきまつび@末尾再帰}とは、関数のすべての条件分岐の末尾が再帰で終わっている再帰のことだ。

例えば以下は階乗を計算する再帰で書かれたループだ。

\ifTombow\pagebreak\fi
\begin{lstlisting}[language={C++}]
int factorial( int n )
{
    if ( n < 1 )
        return 0 ;
    else if ( n == 1 )
        return 1 ;
    else
        return n * factorial(n-1) ;
}
\end{lstlisting}

\texttt{factorial(n)}は\(1 \times 2 \times 3 \times ... \times n\)を計算する。

この関数は、引数\texttt{n}が\texttt{1}未満であれば引数が間違っているので\texttt{0}を返す。そうでない場合で\texttt{n}が\texttt{1}であれば\texttt{1}を返す。それ以外の場合、\texttt{n * factorial(n-1)}を返す。

このコードは末尾再帰になっている。末尾再帰は非再帰のループに機械的に変換できる特徴を持っている。例えば以下のように、
\begin{lstlisting}[language={C++}]
int factorial( int n )
{
    int temp = n ;

loop :
    if ( n < 1 )
        return 0 ;
    else if ( n == 1 )
        return temp * 1 ;
    else
    {
        n = n-1 ;
        temp *= n ;
        goto loop ;
    }
}
\end{lstlisting}
関数のすべての条件分岐の末尾が再帰になっているため、機械的に関数呼び出しを\texttt{goto}文で置き換えることができる。

ただし、プログラミング言語C++の標準規格は、C++の実装に末尾再帰の最適化を義務付けてはいない。そのため、末尾再帰が最適化されるかどうかはC++コンパイラー次第だ。

再帰は強力なループの実現方法で、再帰的な問題を解くのに最適だが、落とし穴もある。
