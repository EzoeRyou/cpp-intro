\hyperchapter{ch11}{整数}{整数}
\index{せいすう@整数}

始めに書いておくがこの章はユーモア欠落症患者によって書かれており極めて退屈だ。しかし、整数の詳細はすべてのプログラマーが理解すべきものだ。心して読むとよい。

\hypersection{ch1101}{整数リテラル}
\index{せいすうりてらる@整数リテラル}

整数リテラルとは整数の値を直接ソースファイルに記述する機能だ。本書ではここまで何の説明もなくリテラルを使っていた。例えば以下のように。

\begin{lstlisting}[language={C++}]
int main()
{
    int a = 123 ;
    int b = 0 ;
    int c = -123 ;
}
\end{lstlisting}

ここでは、\texttt{'123'}, \texttt{'0'}\,がリテラルだ。\texttt{'-123'}\,というのは演算子\texttt{operator -}\,に整数リテラル\texttt{123}を適用したものだ。リテラルは\texttt{123}だけだ。ただしこれは細かい詳細なのでいまはそれほど気にしなくてもよい。

\hypersubsection{ch110101}{10進数リテラル}
\index{10しんすうりてらる@10進数リテラル}\index{せいすうりてらる@整数リテラル!10しんすう@10進数〜}

10進数リテラルは最も簡単で我々が日常的に使っている数の表記方法と同じものだ。接頭語は何も使わず数字には\texttt{0}, \texttt{1}, \texttt{2}, \texttt{3}, \texttt{4}, \texttt{5}, \texttt{6}, \texttt{7}, \texttt{8}, \texttt{9}が使える。

\ifTombow\pagebreak\fi
\begin{lstlisting}[language={C++}]
// 10進数で123
int decimal = 123 ;
\end{lstlisting}

ただし、10進数リテラルの先頭を\texttt{0}にしてはならない。これは8進数リテラルになってしまう。

\begin{lstlisting}[language={C++}]
// 10進数で83
int octal = 0123 ;
\end{lstlisting}

\hypersubsection{ch110102}{2進数リテラル}
\index{2しんすうりてらる@2進数リテラル}\index{せいすうりてらる@整数リテラル!2しんすう@2進数〜}

2進数リテラルは接頭語\texttt{'0b'}, \texttt{'0B'}\,から始まる。数字には\texttt{0}, \texttt{1}を使うことができる。

\begin{lstlisting}[language={C++}]
// 10進数で5
int binary = 0b1010 ;

// 0bと0Bは同じ
int a = 0B1010 ;
\end{lstlisting}

\hypersubsection{ch110103}{8進数リテラル}
\index{8しんすうりてらる@8進数リテラル}\index{せいすうりてらる@整数リテラル!8しんすう@8進数〜}

8進数リテラルは接頭語\,\texttt{'0'}\,から始まる。数字には\texttt{0}, \texttt{1}, \texttt{2}, \texttt{3}, \texttt{4}, \texttt{5}, \texttt{6}, \texttt{7}を使うことができる。

\begin{lstlisting}[language={C++}]
// 10進数で83
int octal = 0123 ;

// 10進数で342391
int a = 01234567 ;
\end{lstlisting}

\hypersubsection{ch110104}{16進数リテラル}
\index{16しんすうりてらる@16進数リテラル}\index{せいすうりてらる@整数リテラル!16しんすう@16進数〜}

16進数リテラルは接頭語\texttt{'0x'}, \texttt{'0X'}\,から始まる。数字には\texttt{0}, \texttt{1}, \texttt{2}, \texttt{3}, \texttt{4}, \texttt{5}, \texttt{6}, \texttt{7}, \texttt{8}, \texttt{9}, \texttt{a}, \texttt{b}, \texttt{c}, \texttt{d}, \texttt{e}, \texttt{f}, \texttt{A}, \texttt{B}, \texttt{C}, \texttt{D}, \texttt{E}, \texttt{F}が使える。ローマ字の大文字と小文字は意味が同じだ。\texttt{a}, \texttt{b}, \texttt{c}, \texttt{d}, \texttt{e}, \texttt{f}がそれぞれ\texttt{10}, \texttt{11}, \texttt{12}, \texttt{13}, \texttt{14}, \texttt{15}を意味する。

\begin{lstlisting}[language={C++}]
// 10進数で291
int hexadecimal = 0x123 ;

// 0xと0Xは同じ
int a = 0X123 ;

// 10進数で10
int b = 0xa ;

// 10進数で15
int c = 0xf ;
\end{lstlisting}

\hypersubsection{ch110105}{数値区切り}
\index{すうちくぎり@数値区切り}\index{せいすうりてらる@整数リテラル!すうちくぎり@数値区切り}

長い整数リテラルは読みにくい。例えば\texttt{10000000}と\texttt{100000000}はどちらが大きくて具体的にどのくらいの値なのかがわからない。C++には整数リテラルを読みやすいように区切ることのできる数値区切りという機能がある。整数リテラルはシングルクオート文字(\,\texttt{'}\,)\index{\protect{'}@\texttt{\protect{'}}}で区切ることができる。

\begin{lstlisting}[language={C++}]
int main()
{
    int a =   (@\textcolor{black}{\texttt{1000'0000}}@) ;
    int b = (@\textcolor{black}{\texttt{1'0000'0000}}@) ;
}
\end{lstlisting}

区切り幅は何文字でもよい。

\begin{lstlisting}[language={C++}]
int main()
{
    int a = (@\textcolor{black}{\texttt{1'22'333'4444'55555}}@) ;
}
\end{lstlisting}

10進数整数リテラル以外でも使える。

\begin{lstlisting}[language={C++}]
int main()
{
    auto a = (@\textcolor{black}{\texttt{0b10101010'11110000'00001111}}@) ;
    auto b = (@\textcolor{black}{\texttt{07'7'5}}@) ;
    auto c = (@\textcolor{black}{\texttt{0xde'ad'be'ef}}@) ;
}
\end{lstlisting}

\hypersection{ch1102}{整数の仕組み}
\index{せいすう@整数!しくみ@仕組み}

\hypersubsection{ch110201}{情報の単位}
\index{じようほうのたんい@情報の単位}

0から100までの整数を表現するには101種類の状態を表現できる必要がある。コンピューターはどうやって整数を表現しているのかをここで学ぶ。

情報の最小単位はビット(bit)\index{びつと@ビット}だ。ビットは2種類の状態を表現できる。たとえば\texttt{bool}型は\texttt{true}/\texttt{false}という2種類の状態を表現できる。

しかし、2種類の状態しか表現できない整数は使いづらい。0もしくは1しか表現できない整数とか、100もしく1000しか表現できない整数は使い物にならない。

また、ビットという単位も扱いづらい。コンピューターは膨大な情報を扱うので、ビットをいくつかまとめたバイト(byte)\index{ばいと@バイト}を単位として情報を扱っている。1バイトが何ビットであるかは環境により異なる。本書では最も普及している1バイトは8ビットを前提にする。

1ビットは2種類の状態を表現できるので、1バイトの中の8ビットは\(2^8 = 256\)種類の状態を表現できる。2バイトならば16ビットとなり、\(2^{16} = 65536\)種類の状態を表現できる。

\hypersubsection{ch110202}{1バイトで表現された整数}

整数の表現方法について理解するために、1バイトで表現された整数を考えよう。

1バイトは8ビットであり256種類の状態を表現できる。整数を0から正の方向の数だけ表現したいとすると、0から255までの値を表現できることになる。

その場合、1バイトの整数の中の8ビットはちょうど2進数8桁で表現できる。

\begin{lstlisting}[language={C++}]
// 0
auto zero = 0b00000000 ;
// 255
auto max  = 0b11111111 ;
\end{lstlisting}

一番左側の桁が最上位桁で、一番右側の桁が最下位桁だ。これを最上位ビット、最下位ビットともいう。

正数だけを表現するならば話は簡単だ。1バイトの整数は0から255までの値を表現できる。これを符号なし整数(unsigned integer)\index{ふごうなしせいすう@符号なし整数}\index{せいすう@整数!ふごうなし@符号なし〜}という。

では負数を表現するにはどうしたらいいだろう。正数と負数を両方扱える整数表現のことを、符号付き整数(signed integer)\index{ふごうつきせいすう@符号付き整数}\index{せいすう@整数!ふごうつき@符号付き〜}という。1バイトは256種類の状態しか表現できないので、もし\(-1\)を表現したい場合、\(-1\)から254までの値を扱えることになる。

\(-1\)しか扱えないのでは実用的ではないので、負数と正数を同じ種類ぐらい表現したい。256の半分は128だが、1バイトで表現された整数は\(-128\)から128までを表現することはできない。0があるからだ。0を含めると、1バイトの整数は最大で\(-128\)から127までか、\(-127\)から128までを表現できる。どちらかに偏ってしまう。

では実際に1バイトで負数も表現できる整数表現を考えてみよう。

\hypersubsubsection{ch11020201}{符号ビット}
\index{ふごうびつと@符号ビット}\index{せいすう@整数!ふごうびつと@符号ビット}

誰でも思いつきそうな表現方法に、符号ビットがある。これは最上位ビットを符号の有無を管理するフラグとして用いることにより、下位7ビットの値の符号を指定する方法だ。

符号ビット表現では\(-1\)と1は以下のように表現できる。

\begin{lstlisting}[language={C++}]
// 1
(@\textcolor{black}{\texttt{0b0'0000001}}@)
// -1
(@\textcolor{black}{\texttt{0b1'0000001}}@)
\end{lstlisting}

最上位ビットが0であれば正数、1であれば負数だ。

この一見わかりやすい表現方法には問題がある。まず表現できる値の範囲は\(-127\)から\(+127\)だ。先ほど、1バイトで正負になるべく均等に値を割り振る場合、\(-128\)から\(+127\)、もしくは\(-127\)から\(+128\)までを扱えると書いた。しかし符号ビット表現では\(-127\)から\(+127\)しか扱えない。残りの1はどこにいったのか。

答えはゼロにある。符号ビット表現ではゼロに2通りの表現がある。\(+0\)と\(-0\)だ。

\begin{lstlisting}[language={C++}]
// +0
(@\textcolor{black}{\texttt{0b0'0000000}}@)
// -0
(@\textcolor{black}{\texttt{0b1'0000000}}@)
\end{lstlisting}

\(+0\)も\(-0\)もゼロには違いない。しかし符号ビットが独立して存在しているために、ゼロが2種類ある。

符号ビットは電子回路で実装するには複雑という問題もある。

\hypersubsubsection{ch11020202}{1の補数}
\index{1のほすう@1の補数}\index{せいすう@整数!1のほすう@1の補数}

1の補数は負数を絶対値を2進数で表したときの各ビットを反転させた値で表現する。たとえば\(-1\)は1(\texttt{0b00000001})の1の補数の\texttt{0b11111110}で表現される。

\begin{lstlisting}[language={C++}]
// -1
0b11111110

// -2
0b11111101
\end{lstlisting}

\(-1\)と\(-2\)を足すと結果は\(-3\)だ。この計算を1の補数で行うとどうなるか。

まず1の補数表現による\(-1\)と\(-2\)を足す。

\begin{lstlisting}[style=terminal]
   11111110
+) 11111101
-----------
 1'11111011
\end{lstlisting}

この結果は9ビットになる。この整数は8ビットなので、9ビット目を表現することはできない。ただし1の補数表現の計算では、もし9ビット目が繰り上がった場合は、演算結果に1を足す取り決めがある。

\begin{lstlisting}[style=terminal]
   11111011
+)        1
-----------
   11111100
\end{lstlisting}

1の補数による\(-3\)は3の各ビットを反転したものだ。3は\texttt{0b00000011}で、そのビットを反転させたものは\texttt{0b11111100}だ。上の計算結果は\(-3\)の1の補数表現になった。

もう1つ例を見てみよう。5と\(-2\)を足すと3になる。

\begin{lstlisting}[style=terminal]
   00000101
+) 11111101
-----------
 1'00000010
\end{lstlisting}

繰り上がりが発生したので1を足すと
\begin{lstlisting}[style=terminal]
   00000010
+)        1
-----------
   00000011
\end{lstlisting}
3になった。

1の補数は引き算も足し算で表現できるので電子回路での実装が符号ビットよりもやや簡単になる。

ただし、1の補数にも問題がある。0の表現だ。0というのは\texttt{0b00000000}だが1の補数では\(-x\)は\(x\)の各ビット反転ということを適用すると、\(-0\)は\texttt{0b11111111}になる。すると、符号ビット表現と同じく、\(+0\)と\(-0\)が存在することになる。したがって、1の補数8ビットで表現できる範囲は\(-127\)から\(+127\)になる。

\hypersubsubsection{ch11020203}{2の補数}
\index{2のほすう@2の補数}\index{せいすう@整数!2のほすう@2の補数}

符号ビットと1の補数による負数表現にある問題は、2の補数表現で解決できる。

2の補数表現による負数は1の補数表現の負数に、繰り上がり時に足すべき1を加えた値になる。

\(-1\)は1の補数表現では、1(\texttt{0b00000001})の各ビットを反転させた値になる(\texttt{0b11111110})。2の補数表現では、1の補数表現に1を加えた値になるので、\texttt{0b11111111}になる。

同様に、\(-2\)は\texttt{0b11111110}に、\(-3\)は\texttt{0b11111101}になる。

2の補数表現の\(-1\)と\(-2\)を足すと以下のようになる。

\begin{lstlisting}[style=terminal]
   11111111
+) 11111110
-----------
 1'11111101
\end{lstlisting}

9ビット目の繰り上がりを無視すると、計算結果は\texttt{0b11111101}になる。これは2の補数表現による\(-3\)と同じだ。

5と\(-2\)の計算も見てみよう。

\ifTombow\pagebreak\fi
\begin{lstlisting}[style=terminal]
   00000101
+) 11111110
-----------
 1'00000011
\end{lstlisting}

結果は3(\texttt{0b00000011})だ。

2の補数表現は引き算も足し算で実装できる上に、ゼロの表現方法は1つで、\(+0\)と\(-0\)が存在しない。8ビットの2の補数表現された整数の範囲は\(-128\)から\(+127\)になる。とても便利な負数の表現方法なのでほとんどのコンピューターで採用されている。

\hypersection{ch1103}{整数型}
\index{せいすうがた@整数型}

C++にはさまざまな整数型が存在する。C++はCから引き継いだ歴史的な経緯により、整数型の文法がわかりにくくなっている。

基本的には、符号付き整数型と符号なし整数型に分かれている。

符号付き整数型\index{ふごうつきせいすうがた@符号付き整数型}としては、\texttt{signed char}, \texttt{short int}, \texttt{int}, \texttt{long int}, \texttt{long long int}が存在する。符号付き整数型は負数を表現できる。

符号なし整数型\index{ふごうなしせいすうがた@符号なし整数型}としては、\texttt{unsigned char}, \texttt{unsigned short int}, \texttt{unsigned int}, \texttt{unsigned long int}, \texttt{unsigned long long int}が存在する。符号なし整数型は負数を表現できない。

\hypersubsection{ch110301}{int型}

\texttt{int型}\index{int@\texttt{int}型}は最も基本となる整数型だ。C++で数値を扱う場合、多くは\texttt{int}型になる。

\begin{lstlisting}[language={C++}]
int x = 123 ;
\end{lstlisting}

整数リテラル\index{せいすうりてらる@整数リテラル}の型は通常は\texttt{int}型になる。

\begin{lstlisting}[language={C++}]
// int
auto x = 123 ;
\end{lstlisting}

\texttt{unsigned int型}\index{unsigned int@\texttt{unsigned int}型}は符号のない\texttt{int}型だ。

\begin{lstlisting}[language={C++}]
unsigned int x = 123 ;
\end{lstlisting}

整数リテラルの末尾に\texttt{u}/\texttt{U}\index{uU@\texttt{u}/\texttt{U}}と書いた場合、\texttt{unsigned int}型になる。

\begin{lstlisting}[language={C++}]
// int
auto x = 123 ;
// unsigned int
auto y = 123u ;
\end{lstlisting}

特殊なルールとして、単に\texttt{signed}と書いた場合、それは\texttt{int}になる。\texttt{unsigned}と書いた場合は、\texttt{unsigned int}になる。

\begin{lstlisting}[language={C++}]
// int
signed a = 1 ;
// unsigned int
unsigned b = 1 ;
\end{lstlisting}

\texttt{signed int}と書いた場合、\texttt{int型}になる。\texttt{signed int}は\texttt{int}の冗長な書き方だ。

\hypersubsection{ch110302}{long int型}

\texttt{long int型}\index{long int@\texttt{long int}型}は\texttt{int型}以上の範囲の整数を扱える型だ。具体的な整数型の値の範囲は実装依存だが、\texttt{long int型}は\texttt{int型}の表現できる整数の範囲はすべて表現でき、かつ\texttt{int型}以上の範囲の整数型を表現できるかもしれない型だ。

\texttt{unsigned long int型}\index{unsigned long int@\texttt{unsigned long int}型}は符号なしの\texttt{long int}だ。

\begin{lstlisting}[language={C++}]
long int a = 123 ;
unsigned long int b = 123 ;
\end{lstlisting}

特殊なルールとして、単に\texttt{long}と書いた場合、それは\texttt{long int}になる。\texttt{unsigned long}と書いた場合、\texttt{unsigned long int}になる。

\begin{lstlisting}[language={C++}]
// long int
long a = 1 ;
// unsigned long int
unsigned long b = 1 ;
\end{lstlisting}

通常、\texttt{int}を省略して単に\texttt{long}と書くことが多い。

整数リテラルの値が\texttt{int型}で表現できない場合、\texttt{long型}になる。例えば、\texttt{int型}で100億を表現できないが、\texttt{long型}では表現できる実装の場合、以下の変数\texttt{a}は\texttt{long型}になる。

\begin{lstlisting}[language={C++}]
// 100億
auto a = (@\textcolor{black}{\texttt{100'0000'0000}}@) ;
\end{lstlisting}

整数リテラルの値が\texttt{long}では表現できないが\texttt{unsigned long}では表現できる場合、\texttt{unsigned long型}になる。

整数リテラルの末尾に\texttt{l}/\texttt{L}\index{lL@\texttt{l}/\texttt{L}}と書いた場合、値にかかわらず\texttt{long型}になる。

\begin{lstlisting}[language={C++}]
// int
auto a = 123 ;
// long
auto b = 123l ;
// long
auto c = 123L ;
\end{lstlisting}

符号なし整数型を意味する\texttt{u}/\texttt{U}と組み合わせることもできる。

\begin{lstlisting}[language={C++}]
// unsigned long
auto a = 123ul ;
auto b = 123lu ;
\end{lstlisting}

順番と大文字小文字の組み合わせは自由だ。

\hypersubsection{ch110303}{long long int型}

\texttt{long long int型}\index{long long int@\texttt{long long int}型}は\texttt{long int型}以上の範囲の整数を扱える型だ。\texttt{long}と同じく\texttt{long long}は\texttt{long long int}と同じで、\texttt{unsigned long long int}\index{unsigned long long int@\texttt{unsigned long long int}型}もある。

\begin{lstlisting}[language={C++}]
// long long int
long long a = 1 ;
// unsigned long long int
unsigned long long b = 1 ;
\end{lstlisting}

整数リテラルの値が\texttt{long型}でも表現できないときは、\texttt{long long}が使われる。\texttt{long long}でも表現できない場合は\texttt{unsigned long long}が使われる。

整数リテラルの末尾に\texttt{ll}/\texttt{LL}\index{llLL@\texttt{ll}/\texttt{LL}}と書くと\texttt{long long int型}になる。

\begin{lstlisting}[language={C++}]
// long long int
auto a = 123ll ;
// long long int
auto b = 123LL ;
// unsigned long long int
auto c = 123ull ;
\end{lstlisting}

\hypersubsection{ch110304}{short int型}

\texttt{short int型}\index{short int@\texttt{short int}型}は\texttt{int型}より小さい範囲の値を扱う整数型だ。\texttt{long}, \texttt{long long}と同様に、\texttt{unsigned short int}型\index{unsigned short int@\texttt{unsigned short int}型}もある。単に\texttt{short}と書くと、\texttt{short int}と同じ意味になる。

整数リテラルで\texttt{short int}型を表現する方法はない。

\hypersubsection{ch110305}{char型}

\texttt{char型}はやや特殊で、\texttt{char}\index{char@\texttt{char}型}, \texttt{signed char}\index{signed char@\texttt{signed char}型}, \texttt{unsigned char}\index{unsigned char@\texttt{unsigned char}型}の3種類の型がある。\texttt{signed char}と\texttt{char}は別物だ。\texttt{char型}は整数型であり、あとで説明するように文字型でもある。\texttt{char型}の符号の有無は実装ごとに異なる。

\hypersection{ch1104}{整数型のサイズ}

整数型を含む変数のサイズ\index{せいすうがた@整数型!さいず@サイズ}\index{へんすう@変数!さいず@サイズ}は、\texttt{sizeof演算子}\index{sizeof@\texttt{sizeof}演算子}で確認することができる。\texttt{sizeof(T)}は\texttt{T}に型名や変数名を入れることで、サイズを取得することができる。

\begin{lstlisting}[language={C++}]
int main()
{
    std::cout << sizeof(int) << "\n"s ;

    int x{} ;
    std::cout << sizeof(x) ;
}
\end{lstlisting}

\texttt{sizeof演算子}は\texttt{std::size\_t型}\index{size\_t@\texttt{size\_t}型}を返す。\texttt{vector}の章でも出てきたこの型は実装依存の符号なし型であると定義されている。単位はバイトだ。

以下が各種整数型のサイズを出力するプログラムだ。

\begin{lstlisting}[language={C++}]
int main()
{
    auto print = []( std::size_t s )
    { std::cout << s << "\n"s ; } ;

    print( sizeof(char) ) ;
    print( sizeof(short) ) ;
    print( sizeof(int) ) ;
    print( sizeof(long) ) ;
    print( sizeof(long long ) ) ;
}
\end{lstlisting}

このプログラムを筆者の環境で実行した結果が以下になる。

\begin{lstlisting}[style=terminal]
1
2
4
8
8
\end{lstlisting}

どうやら筆者の環境では、\texttt{char}が1バイト、\texttt{short}が2バイト、\texttt{int}が4バイト、\texttt{long}と\texttt{long long}が8バイトのようだ。この結果は環境ごとに異なるので読者も自分で\texttt{sizeof}演算子をさまざまな型に適用して試してほしい。

\hypersection{ch1105}{整数型の表現できる値の範囲}

整数型の表現できる値の最小値\index{せいすうがた@整数型!さいしようち@最小値}と最大値\index{せいすうがた@整数型!さいだいち@最大値}は\texttt{std::numeric\_limits<T>}\,\index{numeric\_limits@\texttt{numeric\_limits}}で取得できる。最小値は\texttt{::min()}\index{min@\texttt{min}}を、最大値は\texttt{::max()}\index{max@\texttt{max}}で得られる。

\begin{lstlisting}[language={C++}]
int main()
{
    std::cout
        << std::numeric_limits<int>::min() << "\n"s
        << std::numeric_limits<int>::max() ;
}
\end{lstlisting}

実行結果

\begin{lstlisting}[style=terminal]
-2147483648
2147483647
\end{lstlisting}

どうやら筆者の環境では\texttt{int}型は\(−21億4748万3648\)から21億4748万3647までの範囲の値を表現できるようだ。

\texttt{unsigned int}はどうだろうか。

\begin{lstlisting}[language={C++}]
int main()
{
    std::cout
        << std::numeric_limits<unsigned int>::min() << "\n"s
        << std::numeric_limits<unsigned int>::max() ;
}
\end{lstlisting}

実行結果

\begin{lstlisting}[style=terminal]
0
4294967295
\end{lstlisting}

どうやら筆者の環境では\texttt{unsigned int}型は0から42億9496万7295までの範囲の値を表現できるようだ。\texttt{sizeof(int)}が4バイトであり、1バイトが8ビットの筆者の環境では自然な値だ。符号なしの4バイト整数型は0から\(2^{32}-1\)までの範囲の値を表現できる。符号付き4バイト整数型は\(-2^{31}\)から\(2^{31}-1\)までの範囲の値を表現できる。

整数の最小値を\(-1\)したり、最大値を\(+1\)した場合、何が起こるのだろうか。

符号なし整数型の場合は簡単だ。最小値\(-1\)は最大値になる。最大値\(+1\)は最小値になる。

\ifTombow\pagebreak\fi
\begin{lstlisting}[language={C++}]
int main()
{
    unsigned int min = std::numeric_limits<unsigned int>::min() ;
    unsigned int max = std::numeric_limits<unsigned int>::max() ;

    unsigned int min_minus_one = min - 1u ;
    unsigned int max_plus_one = max + 1u ;

    std::cout << min << "\n"s << max << "\n"s
        << min_minus_one << "\n"s << max_plus_one ;
}
\end{lstlisting}

8ビットの符号なし整数型があるとして、最小値は\texttt{0b00000000}(0)になるが、この値を\(-1\)すると\texttt{0b11111111}(255)となり、これは最大値になる。逆に、最大値である\texttt{0b11111111}(255)に\(+1\)すると\texttt{0b00000000}(0)となり、これは最小値になる。

これを数学的に厳密に書くと、「符号なし整数は算術モジュロ\(2^n\)の法に従う。ただし\(n\)は整数を表現する値のビット数である」となる。

符号付き整数型の場合、挙動は定められていない。ただし、一般に普及している2の補数表現の場合は、以下のような挙動になることが多い。

符号付き整数型の最小値を\(-1\)すると最大値になり、最大値を\(+1\)すると最小値になる。

\begin{lstlisting}[language={C++}]
int main()
{
     int min = std::numeric_limits<int>::min() ;
     int max = std::numeric_limits<int>::max() ;

     int min_minus_one = min - 1 ;
     int max_plus_one = max + 1 ;

    std::cout << min << "\n"s << max << "\n"s
        << min_minus_one << "\n"s << max_plus_one ;
}
\end{lstlisting}

これはなぜか。2の補数表現の8ビットの符号付き整数の最小値は\texttt{0b10000000}(\(-128\))だが、これを\(-1\)すると\texttt{0b01111111}(127)となり、これは最大値となる。逆に最大値\texttt{0b01111111}(127)を\(+1\)すると\texttt{0b10000000}(\(-128\))となり、これは最小値となる。

\clearpage
\hypersection{ch1106}{整数型の変換}
\index{せいすうがた@整数型!へんかん@変換}

整数型にはここで紹介しただけでも、さまざまな型がある。同じ型同士を使った方がよい。

以下は型が一致している例だ。

\begin{lstlisting}[language={C++}]
int main()
{
    int a = 123 ;
    long b = 123l ;
    long long c = 123ll ;

    unsigned int d = 123u ; 
}
\end{lstlisting}

以下は型が一致していない例だ。

\begin{lstlisting}[language={C++}]
int main()
{
    // intからshort
    short a = 123 ;
    // longからint
    int b = 123l ;

    // intからunsigned int
    unsigned int c = 123 ;
    // unsigned intからint
    int d = 123u ;
}
\end{lstlisting}

代入や演算で整数型が一致しない場合、整数型の変換が行われる。

整数型の変換で注意すべきこととしては、変換元の値を変換先の型で表現できない場合の挙動だ。

たとえば\texttt{short}型と\texttt{int}型の表現できる最大値を調べるプログラムを書いてみよう。

\begin{lstlisting}[language={C++}]
int main()
{
    std::cout << "short: "s << std::numeric_limits<short>::max() << "\n"s
        << "int: "s << std::numeric_limits<int>::max() ;
}
\end{lstlisting}

これを実行すると筆者の環境では以下のようになる。

\begin{lstlisting}[style=terminal]
short: 32767
int: 2147483647
\end{lstlisting}

どうやら筆者の環境では\texttt{short}型は約3万、\texttt{int}型は約21億ぐらいの値を表現できるようだ。

では約3万までしか表現できない\texttt{short}型に4万を代入しようとするとどうなるのか。これは1つ前の整数型の表現できる値の範囲で説明したものと同じことが起こる。

\begin{lstlisting}[language={C++}]
int main()
{
    short x = 40000 ;
    std::cout << x ;
}
\end{lstlisting}

このプログラムを実行した結果は実装ごとに異なる。例えば筆者の環境では以下のようになる。

\begin{lstlisting}[style=terminal]
-25536
\end{lstlisting}

整数型の変換は暗黙的に行われるが、明示的に行うこともできる。明示的な変換には\texttt{static\_cast<T>(e)}\index{static\_cast@\texttt{static\_cast}}を使う。\texttt{static\_cast}は値\texttt{e}を型\texttt{T}の値に変換する。

\begin{lstlisting}[language={C++}]
int main()
{
    int x = 123 ;
    short y = static_cast<short>(x) ;
}
\end{lstlisting}

