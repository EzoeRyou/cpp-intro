\hyperchapter{ch15}{lvalueリファレンスとconst}{lvalueリファレンスとconst}

\begin{quote}
\begin{flushright}
ポップカルチャーリファレンスというのは\\
流行の要素をさり気なく作品中に取り入れることで、\\
流行作品を知っている読者の笑いを誘う手法である\\
--- キャプテン・オブビウス、ポップカルチャーリファレンスについて
\end{flushright}
\end{quote}

\hypersection{ch1501}{lvalueリファレンス}
\index{lvalue@\texttt{lvalue}リファレンス}

変数に変数を代入すると、代入元の値が代入先にコピーされる。代入先の値を変更しても、コピーされた値が変わるだけで、代入元にはいっさい影響がない。

\begin{lstlisting}[language={C++}]
int main()
{
    int a = 1 ;
    int b = 2 ;

    b = a ;
    // b == 1

    b = 3 ;
    // a == 1
    // b == 3
}
\end{lstlisting}

\ifTombow\pagebreak\fi
これは関数も同じだ。

\begin{lstlisting}[language={C++}]
void assign_3( int x )
{
    x = 3 ;
}

int main()
{
    int a = 1 ;
    assign_3( a ) ;

    // a == 1
}
\end{lstlisting}

しかし、ときには変数の値を直接書き換えたい場合がある。このとき\texttt{lvalue}リファレンス(reference)が使える。\texttt{lvalue}リファレンスは変数に\,\texttt{\&}\,\index{\&@\texttt{\&}}を付けて宣言する

\begin{lstlisting}[language={C++}]
int main()
{
    int a = 1 ;
    int & ref = a ;

    ref = 3 ;

    // a == 3
    // refはaなので同じく3
}
\end{lstlisting}

この例で、変数\texttt{ref}は変数\texttt{a}への参照(リファレンス)\index{さんしよう@参照}\index{りふあれんす@リファレンス}なので、変数\texttt{a}と同じように使える。

\texttt{lvalue}リファレンスは必ず初期化しなければならない。

\begin{lstlisting}[language={C++}]
int main()
{
    // エラー
    int & ref ;
}
\end{lstlisting}

\texttt{lvalue}リファレンスは関数でも使える。

\begin{lstlisting}[language={C++}]
void f( int & x )
{
    x = 3 ;
}

int main()
{
    int a = 1 ;
    f( a ) ;

    // a == 3
}
\end{lstlisting}

選択ソートで2つの変数の値を交換する必要があったことを覚えているだろうか。

\begin{lstlisting}[language={C++}]
int main()
{
    std::vector<int> v = {3,2,1,4,5} ;

    // 0番目と2番目の要素を交換したい
    auto temp = v.at(0) ;
    v.at(0) = v.at(2) ;
    v.at(2) = temp ;
}
\end{lstlisting}

いちいち交換のために別の変数\texttt{temp}を作って3回代入を書くのは面倒だ。これを関数にしてしまいたい。

\begin{lstlisting}[language={C++}]
// 値を交換
swap( v.at(0), v.at(2) ) ;
\end{lstlisting}

このような関数\texttt{swap}は普通に書くことはできない。

\begin{lstlisting}[language={C++}]
// この実装は正しくない
auto swap = []( auto a, auto b )
{
    auto temp = a ;
    a = b ;
    b = temp ;
} ;
\end{lstlisting}

この実装では、変数は単にコピーされるだけなので、関数の呼び出し元には何の影響もない。

これを\texttt{lvalue}リファレンスに変えると、関数の呼び出し元の変数の値を交換する関数\texttt{swap}が作れる。

\ifTombow\pagebreak\fi
\begin{lstlisting}[language={C++}]
// lvalueリファレンス
auto swap = []( auto & a, auto & b )
{
    auto temp = a ;
    a = b ;
    b = temp ;
} ;
\end{lstlisting}

C++の標準ライブラリには\texttt{std::swap}\index{swap@\texttt{swap}}があるので、読者はわざわざこのような関数を自作する必要はない。

\begin{lstlisting}[language={C++}]
int main()
{
    int a = 1 ;
    int b = 2 ;

    std::swap( a, b ) ;

    // a == 2
    // b == 1
}
\end{lstlisting}

ところで、この章では一貫して\texttt{lvalue}リファレンスと書いているのに気が付いただろうか。\texttt{lvalue}とは何なのか、\texttt{lvalue}ではないリファレンスはあるのか。その疑問はあとの章で解決する。

\hypersection{ch1502}{const}
\index{const@\texttt{const}}

値を変更したくない変数は、\texttt{const}を付けることで変更を禁止できる。

\begin{lstlisting}[language={C++}]
int main()
{
    int x = 0 ;
    x = 1 ; // OK、変更できる

    const int y = 0 ;
    y = 0 ; // エラー、 変更できない。
}
\end{lstlisting}

\texttt{const}はちょっと文法が変わっていて混乱する。例えば、\texttt{const int}でも\texttt{int const}でも意味が同じだ。

\ifTombow\pagebreak\fi
\begin{lstlisting}[language={C++}]
int main()
{
    // 意味は同じ
    const int x = 0 ;
    int const y = 0 ;
}
\end{lstlisting}

\texttt{const}は\texttt{lvalue}リファレンスと組み合わせることができる。

\begin{lstlisting}[language={C++}]
int main()
{
    int x = 0 ;

    int & ref = x ;
    // OK
    ++ref ;

    const int & const_ref = x ;

    // エラー
    ++const_ref ;
}
\end{lstlisting}

\texttt{const}は本当に文法が変わっていて混乱する。\texttt{const int \&}と\texttt{int const \&}は同じ意味だが、\texttt{int \& const}はエラーになる。

\begin{lstlisting}[language={C++}]
int main()
{
    int a = 0 ;

    // OK、意味は同じ
    const int & b = a ;
    int const & c = a ;

    // エラー
    int & const d = a ;
}
\end{lstlisting}

これはとても複雑なルールで決まっているので、こういうものだとあきらめて覚えるしかない。

\texttt{const}が付いていない型のオブジェクトを\texttt{const}な\texttt{lvalue}リファレンスで参照することができる。

\ifTombow\pagebreak\fi
\begin{lstlisting}[language={C++}]
int main()
{
    // constの付いていない型のオブジェクト
    int x = 0 ;

    // OK
    int & ref = x ;
    // OK、constは付けてもよい
    const int & cref = x ;
}
\end{lstlisting}

\texttt{const}の付いている型のオブジェクトを\texttt{const}の付いていない\texttt{lvalue}リファレンスで参照することはできない。

\begin{lstlisting}[language={C++}]
int main()
{
    // constの付いている型のオブジェクト
    const int x = 0 ;

    // エラー、 constがない
    int & ref = x ;

    // OK、constが付いている
    const int & cref = x ;
}
\end{lstlisting}

\texttt{const}の付いている\texttt{lvalue}リファレンスは何の役に立つのかというと、関数の引数を取るときに役に立つ。

例えば以下のコードは非効率的だ。

\begin{lstlisting}[language={C++}]
void f( std::vector<int> v )
{
    std::cout << v.at(1234) ;
}

int main()
{
    // 10000個の要素を持つvector
    std::vector<int> v(10000) ;

    f( v ) ;
}
\end{lstlisting}

なぜかというと、関数の引数に渡すときに、変数\texttt{v}はコピーされるからだ。

リファレンスを使うと不要なコピーをしなくて済む。

\begin{lstlisting}[language={C++}]
void f( std::vector<int> & v )
{
    std::cout << v.at(1234) ;
}
\end{lstlisting}

しかし、リファレンスで受け取ると、うっかり変数を変更してしまった場合、その変更が関数の呼び出し元に反映されてしまう。

\begin{lstlisting}[language={C++}]
// 値を変更するかもしれない
void f( std::vector<int> & v ) ;

int main()
{
    // 要素数10000のvector
    std::vector<int> v(10000) ;

    f(v) ;

    // 値は変更されているかもしれない
}
\end{lstlisting}

このとき、\texttt{const}な\texttt{lvalue}リファレンスを使うと、引数に取った値を変更しないことを保証できる。

\begin{lstlisting}[language={C++}]
void f( std::vector<int> const & v ) ;
\end{lstlisting}

