\hyperchapter{ch01}{C++の概要}{C++の概要}

C++\index{C++}とは何か。C++の原作者にして最初の実装者であるBjarne Stroustrup\index{Stroustrup, Bjarne}は、以下のように簡潔にまとめている。

\begin{quote}
C++は、Simulaのプログラム構造化のための機構と、Cのシステムプログラミング用の効率性と柔軟性を提供するために設計された。C++は半年ほどで現場で使えることを見込んでいた。結果として成功した。
\begin{flushright}
Bjarne Stroustrup, A History of C++: 1979--1991, HOPL2
\end{flushright}
\end{quote}

プログラミング言語史に詳しくない読者は、Simula\index{Simula}というプログラミング言語について知らないことだろう。Simulaというのは、初めてオブジェクト指向プログラミングを取り入れたプログラミング言語だ。当時と言えばまだ高級なプログラミング言語はほとんどなく、\texttt{if else}, \texttt{while}などのIBMの提唱した構造化プログラミング\index{こうぞうかぷろぐらみんぐ@構造化プログラミング}を可能にする文法を提供しているプログラミング言語すら、多くは研究段階であった。いわんやオブジェクト指向など、当時はまだアカデミックにおいて可能性の1つとして研究されている程度の地に足のついていない夢の機能であった。そのような粗野な時代において、Simulaは先進的なオブジェクト指向プログラミング\index{おぶじえくとしこうぷろぐらみんぐ@オブジェクト指向プログラミング}を実現していた。

オブジェクト指向は現代のプログラミング言語ではすっかり普通になった。データの集合とそのデータに適用する関数を関連付けることができる便利なシンタックスシュガー、つまりプログラミング言語の文法上の機能として定着した。しかし、当時のオブジェクト指向というのはもっと抽象度の高い概念であった。本来のオブジェクト指向をプログラミング言語に落とし込んだ最初の言語として、SimulaとSmalltalkがある。

Simula\index{Simula}ではクラスのオブジェクト1つ1つが、あたかも並列実行しているかのように振る舞った。Smalltalkでは同一プログラム内のオブジェクトごとのデータのやり取りですらあたかもネットワーク越しに通信をするかのようなメッセージパッシングで行われた。

問題は、そのような抽象度の高すぎるSimulaやSmalltalkのようなプログラミング言語の設計と実装では実行速度が遅く、大規模なプログラムを開発するには適さなかった。

Cの効率性と柔軟性というのは、要するに実行速度が速いとかメモリー消費量が少ないということだ。ではなぜCはほかの言語に比べて効率と柔軟に優れているのか。これには2つの理由がある。

1つ、Cのコードは直接ハードウェアがサポートする命令にまでマッピング可能であるということ。現実のハードウェアにはストレージがあり、メモリーがあり、キャッシュがあり、レジスターがあり、命令は投機的に並列実行される泥臭い計算機能を提供している。

1つ、使わない機能のコストを支払う必要がないというゼロオーバーヘッドの原則。例えばあらゆるメモリー利用がGC(ガベージコレクション)\index{GC@GC(ガベージコレクション)}によって管理されている言語では、たとえメモリーをすべて明示的に管理していたとしても、GCのコストを支払わなければならない。GCではプログラマーは確保したメモリーの解放処理を明示的に書く必要はない。定期的に全メモリーを調べて、どこからも使われていないメモリーを解放する。この処理には余計なコストがかかる。しかし、いつメモリーを解放すべきかがコンパイル時に決定できる場合では、GCは必要ない。GCが存在する言語では、たとえGCが必要なかったとしても、そのコストを支払う必要がある。また実行時にメモリーレイアウトを判定して実行時に分岐処理ができる言語では、たとえコンパイル時にメモリーレイアウトが決定されていたとしても、実行時にメモリーレイアウトを判定して条件分岐するコストを支払わなければならない。

C++は、「アセンブリ言語をおいて、C++より下に言語を置かない」と宣言するほど、ハードウェア機能への直接マッピング\index{ちよくせつまつぴんぐ@直接マッピング}とゼロオーバーヘッド\index{ぜろおばへつど
@ゼロオーバーヘッド}の原則を重視している。

C++のほかの特徴としては、委員会方式による国際標準規格\index{こくさいひようじゆんきかく@国際標準規格}を定めていることがある。特定の一個人や一法人が所有する言語は、個人や法人の意思で簡単に仕様が変わってしまう。短期的な利益を追求するために長期的に問題となる変更をしたり、単一の実装が仕様だと言わんばかりの振る舞いをする。特定の個人や法人に所有されていないこと、実装が従うべき標準規格があること、独立した実装が複数あること、言語に利害関係を持つ関係者が議論して投票で変更を可決すること、これがC++が長期に渡って使われてきた理由でもある。

委員会方式の規格制定では、下位互換性の破壊は忌避される。なぜならば、既存の動いているコードを壊すということは、それまで存在していた資産の価値を毀損することであり、利害関係を持つ委員が反対するからだ。

下位互換性を壊した結果何が起こるかというと、単に言語が新旧2つに分断される。Python 2とPython 3がその最たる例だ。

C++には今日の最新で高級な言語からみれば古風な制約が数多く残っているが、いずれも理由がある。下位互換性を壊すことができないという理由。効率的な実装方法が存在しないという理由。仮に効率的な実装が存在するにしても、さまざまな環境で実装可能でなければ規格化はできないという理由。

C++には善しあしがある。Bjarne StroustrupはC++への批判にこう答えている。

\begin{quote}
言語には2種類ある。文句を言われる言語と、誰も使わない言語。
\end{quote}

C++は文句を言われる方の言語だ。
