\hyperchapter{ch31}{vectorの実装:メモリー確保}{vectorの実装:メモリー確保}
\index{vector@\texttt{vector}!めもりかくほ@メモリー確保}

\hypersection{ch3101}{メモリー確保と解放の起こるタイミング}

\texttt{std::vector}はどこでメモリーを確保と解放しているのだろうか。

デフォルト構築すると空になる。

\begin{lstlisting}[language={C++}]
int main()
{
    std::vector<int> v ;
    v.empty() ; // true
}
\end{lstlisting}

コンストラクターに要素数を渡すことができる。
\index{vector@\texttt{vector}!ようそすう@要素数}

\begin{lstlisting}[language={C++}]
int main()
{
    std::vector<int> v(100) ;
    v.size() ; // 100
}
\end{lstlisting}

すると\texttt{std::vector}は指定した要素数の有効な要素を持つ。

コンストラクターに要素数と初期値を渡すことができる。

\begin{lstlisting}[language={C++}]
int main()
{
    std::vector<int> v(100, 123) ;
    v[0] ; // 123
    v[12] ; // 123
    v[68] ; // 123
}
\end{lstlisting}

すると、指定した要素数で、要素の値はすべて初期値になる。

\texttt{vector}のオブジェクトを構築したあとでも、メンバー関数\texttt{resize(size)}で要素数を\texttt{size}個にできる。
\index{resize@\texttt{resize}}\index{vector@\texttt{vector}!resize@\texttt{resize}}

\begin{lstlisting}[language={C++}]
int main()
{
    std::vector<int> v ;
    v.resize(10) ;
    v.size() ; // 10
    // 減らす
    v.resize(5) ;
    v.size() ; // 5
}
\end{lstlisting}

\texttt{resize}で要素数が増える場合、増えた要素の初期値も指定できる。

\begin{lstlisting}[language={C++}]
int main()
{
    std::vector<int> v ;
    v.resize(3, 123) ;
    // vは{123,123,123}
}
\end{lstlisting}

\texttt{resize}で要素数が減る場合、末尾が削られる。

\begin{lstlisting}[language={C++}]
int main()
{
    std::vector<int> v = {1,2,3,4,5} ;
    v.resize(3) ;
    // vは{1,2,3}
}
\end{lstlisting}

メンバー関数\texttt{push\_back(value)}を呼び出すと要素数が1増え、要素の末尾の要素が値\texttt{value}になる。
\index{push\_back@\texttt{push\_back}}\index{vector@\texttt{vector}!push\_back@\texttt{push\_back}}

\begin{lstlisting}[language={C++}]
int main()
{
    std::vector<int> v ;
    // vは{}
    v.push_back(1) ;
(@\ifTombow\pagebreak\fi@)
    // vは{1}
    v.push_back(2) ;
    // vは[1,2}
    v.push_back(3) ;
    // vは{1,2,3}
}
\end{lstlisting}

\texttt{reserve(size)}は少なくとも\texttt{size}個の要素が追加の動的メモリー確保なしで追加できるようにメモリーを予約する。
\index{reserve@\texttt{reserve}}\index{vector@\texttt{vector}!reserve@\texttt{reserve}}

\begin{lstlisting}[language={C++}]
int main()
{
    std::vector<int> v ;
    // 少なくとも3個の要素を追加できるように動的メモリー確保
    v.reserve(3) ;
    v.size() ; // 0
    v.capacity() ; // 3以上

    // 動的メモリー確保は発生しない
    v.push_back(1) ;
    v.push_back(2) ;
    v.push_back(3) ;
    // 動的メモリー確保が発生する可能性がある。
    v.push_back(3) ;
}
\end{lstlisting}

\texttt{clear()}は要素数を0にする。
\index{clear@\texttt{clear}}\index{vector@\texttt{vector}!clear@\texttt{clear}}

\begin{lstlisting}[language={C++}]
int main()
{
    std::vector<int> v = {1,2,3} ;
    v.clear() ;
    v.size() ; // 0
}
\end{lstlisting}

この章ではここまでの実装をする。

\clearpage
\hypersection{ch3102}{デフォルトコンストラクター}

簡易\texttt{vector}のデフォルトコンストラクターは何もしない。

\begin{lstlisting}[language={C++}]
vector( ) { }
\end{lstlisting}

何もしなくてもポインターはすべて\texttt{nullptr}で初期化され、アロケーターもデフォルト構築されるからだ。

これで簡易\texttt{vector}の変数を作れるようになった。ただしまだ何もできない。

\begin{lstlisting}[language={C++}]
int main()
{
    vector<int> v ;
    // まだ何もできない。
}
\end{lstlisting}

\hypersection{ch3103}{アロケーターを取るコンストラクター}
\index{こんすとらくた@コンストラクター}\index{vector@\texttt{vector}!こんすとらくた@コンストラクター}

\texttt{std::vector}のコンストラクターは最後の引数にアロケーターを取れる。

\begin{lstlisting}[language={C++}]
int main()
{
    std::allocator<int> alloc ;
    // 空
    std::vector<int> v1(alloc) ;
    // 要素数5
    std::vector<int> v2(5, alloc) ;
    // 要素数5で初期値123
    std::vector<int> v3(5, 123, alloc) ;
}
\end{lstlisting}

これを実装するには、アロケーターを取ってデータメンバーにコピーするコンストラクターを書く。

\begin{lstlisting}[language={C++}]
vector( const allocator_type & alloc ) noexcept
    : alloc( alloc )
{ }
\end{lstlisting}

ほかのコンストラクターはこのコンストラクターにまずデリゲートすればよい。

\begin{lstlisting}[language={C++}]
vector()
    : vector( allocator_type() )
{ }

vector( size_type size, const allocator_type & alloc = allocator_type() )
    : vector( alloc )
{ /* 実装 */ }
vector( size_type size, const_reference value, const allocator_type & alloc = allocator_type() )
    : vector( alloc )
{ /* 実装 */ }
\end{lstlisting}

\hypersection{ch3104}{要素数と初期値を取るコンストラクターの実装}
\index{resize@\texttt{resize}}\index{vector@\texttt{vector}!resize@\texttt{resize}}

要素数と初期値を取るコンストラクターは\texttt{resize}を使えば簡単に実装できる。

\begin{lstlisting}[language={C++}]
vector( size_type size, const allocator_type & alloc )
    : vector( alloc )
{
    resize( size ) ;
}
vector( size_type size, const_reference value, const allocator_type & alloc )
    : vector( alloc )
{
    resize( size, value ) ;
}
\end{lstlisting}

しかしこれは実装を\texttt{resize}に丸投げしただけだ。\texttt{resize}の実装をする前に、実装を楽にするヘルパー関数を実装する。

\hypersection{ch3105}{ヘルパー関数}
\index{へるぱかんすう@ヘルパー関数}\index{vector@\texttt{vector}!へるぱかんすう@ヘルパー関数}

ここでは\texttt{vector}の実装を楽にするためのヘルパー関数をいくつか実装する。このヘルパー関数はユーザーから使うことは想定しないので、\texttt{private}メンバーにする。

\begin{lstlisting}[language={C++}]
// 例
struct vector
{
private :
    // ユーザーからは使えないヘルパー関数
    void helper_function() ;
public :
    // ユーザーが使える関数
    void func()
    {
        // ヘルパー関数を使って実装
        helper_function() ;
    }
} ;
\end{lstlisting}

\hypersubsection{ch310501}{ネストされた型名traits}
\index{あろけた@アロケーター}\index{allocator\_traits@\texttt{allocator\_traits}}\index{vector@\texttt{vector}!allocator\_traits@\texttt{allocator\_traits}}

アロケーターは\texttt{allocator\_traits}を経由して使う。実際のコードはとても冗長になる。

\begin{lstlisting}[language={C++}]
template < typename Allocator >
void f( Allocator & alloc )
{
    std::allocator_traits<Allocator>::allocate( alloc, 1 ) ;
}
\end{lstlisting}

この問題はエイリアス名を使えば解決できる。

\begin{lstlisting}[language={C++}]
private :
    using traits = std::allocator_traits<allocator_type> ;

    template < typename Allocator >
    void f( Allocator & alloc )
    {
        traits::allocate( alloc, 1 ) ;
    }
\end{lstlisting}

\hypersubsection{ch310502}{allocate/deallocate}
\index{allocate@\texttt{allocate}}\index{vector@\texttt{vector}!allocate@\texttt{allocate}}\index{deallocate@\texttt{deallocate}}\index{vector@\texttt{vector}!deallocate@\texttt{deallocate}}

\texttt{allocate(n)}はアロケーターから\texttt{n}個の要素を格納できる生のメモリーの動的確保をして先頭要素へのポインターを返す。

\texttt{deallocate(ptr)}はポインター\texttt{ptr}を解放する。

\begin{lstlisting}[language={C++}]
private: 
    pointer allocate( size_type n )
    { return traits::allocate( alloc, n ) ; }
    void deallocate( )
    { traits::deallocate( alloc, first, capacity() ) ; }
\end{lstlisting}

\hypersubsection{ch310503}{construct/destroy}
\index{construct@\texttt{construct}}\index{vector@\texttt{vector}!construct@\texttt{construct}}\index{destroy@\texttt{destroy}}\index{vector@\texttt{vector}!destroy@\texttt{destroy}}

\texttt{construct(ptr)}は生のメモリーへのポインター\texttt{ptr}に\texttt{vector}の\texttt{value\_type}型の値をデフォルト構築する。

\texttt{construct(ptr, value)}は生のメモリーへのポインター\texttt{ptr}に値\texttt{value}のオブジェクトを構築する。

\ifTombow\pagebreak\fi
\begin{lstlisting}[language={C++}]
    void construct( pointer ptr )
    { traits::construct( alloc, ptr ) ; }
    void construct( pointer ptr, const_reference value )
    { traits::construct( alloc, ptr, value ) ; }
    // ムーブ用
    void construct( pointer ptr, value_type && value )
    { traits::construct( alloc, ptr, std::move(value) ) ; }
\end{lstlisting}

ムーブ用の\texttt{construct}についてはまだ気にする必要はない。この理解には、まずムーブセマンティクスを学ぶ必要がある。

\texttt{destroy(ptr)}は\texttt{ptr}の指すオブジェクトを破棄する。

\begin{lstlisting}[language={C++}]
private :
    void destroy( pointer ptr )
    { traits::destroy( alloc, ptr ) ; }
\end{lstlisting}

\hypersubsection{ch310504}{destroy\texttt{\_}until}
\index{destroy\_until@\texttt{destroy\_until}}\index{vector@\texttt{vector}!destroy\_until@\texttt{destroy\_until}}

\texttt{destroy\_until(rend)}は、\texttt{vector}が保持する\texttt{rbegin()}からリバースイテレーター\texttt{rend}までの要素を破棄する。リバースイテレーターを使うので、要素の末尾から先頭に向けて順番に破棄される。なぜ末尾から先頭に向けて要素を破棄するかというと、C++では値の破棄は構築の逆順で行われるという原則があるからだ。

\begin{lstlisting}[language={C++}]
private :
    void destroy_until( reverse_iterator rend )
    {
        for ( auto riter = rbegin() ; riter != rend ; ++riter, --last )
        {
            destroy( &*riter ) ;
        }
    }
\end{lstlisting}

\texttt{\&*riter}はやや泥臭い方法だ。簡易\texttt{vector<T>}\,の\texttt{iterator}は単なる\texttt{T *}\,だが、\texttt{riter}はリバースイテレーターなのでポインターではない。ポインターを取るために\,\texttt{*riter}でまず\texttt{T \&}を得て、そこに\,\texttt{\&}\,を適用することで\texttt{T *}\,を得ている。

破棄できたら有効な要素数を減らすために\,\texttt{{-}{-}last}する。

\clearpage
\hypersection{ch3106}{clear}
\index{clear@\texttt{clear}}\index{vector@\texttt{vector}!clear@\texttt{clear}}

\texttt{clear()}はすべての要素を破棄する。

\begin{lstlisting}[language={C++}]
void clear() noexcept
{
    destroy_until( rend() ) ;
}
\end{lstlisting}

先ほど実装した\texttt{destroy\_until(rend)}にリバースイテレーターの終端を渡せばすべての要素が破棄される。

\hypersection{ch3107}{デストラクター}
\index{ですとらくた@デストラクター}\index{vector@\texttt{vector}!ですとらくた@デストラクター}

ヘルパー関数を組み合わせることでデストラクターが実装できるようになった。

\texttt{std::vector}のデストラクターは、

\begin{enumerate}
\def\labelenumi{\arabic{enumi}.}
\item
  要素を末尾から先頭に向かう順番で破棄
\item
  生のメモリーを解放する
\end{enumerate}

この2つの処理はすでに実装した。デストラクターの実装は単にヘルパー関数を並べて呼び出すだけでよい。

\begin{lstlisting}[language={C++}]
~vector()
{
    // 1. 要素を末尾から先頭に向かう順番で破棄
    clear() ;
    // 2. 生のメモリーを解放する
    deallocate() ;    
}         
\end{lstlisting}

\hypersection{ch3108}{reserveの実装}
\index{reserve@\texttt{reserve}}\index{vector@\texttt{vector}!reserve@\texttt{reserve}}

\texttt{reserve}の実装は生の動的メモリーを確保してデータメンバーを適切に設定する。

ただし、いろいろと考慮すべきことが多い。

現在の\texttt{capacity}より小さい要素数が\texttt{reserve}された場合、無視してよい。

\begin{lstlisting}[language={C++}]
int main()
{
    // 要素数5
    std::vector<int> v = {1,2,3,4,5} ;
(@\ifTombow\pagebreak\fi@)
    // 3個の要素を保持できるよう予約
    v.reserve( 3 ) ;
    // 無視する
}
\end{lstlisting}

すでに指定された要素数以上に予約されているからだ。

動的メモリー確保が行われていない場合、単に動的メモリー確保をすればよい。

\begin{lstlisting}[language={C++}]
int main()
{
    std::vector<int> v ;
    // おそらく動的メモリー確保
    v.reserve( 10000 ) ;
}
\end{lstlisting}

「おそらく」というのは、C++の規格は\texttt{vector}のデフォルトコンストラクターが予約するストレージについて何も言及していないからだ。すでに要素数10000を超えるストレージが予約されている実装も規格準拠だ。本書で実装している\texttt{vector}は、デフォルトコンストラクターでは動的メモリー確保をしない実装になっている。

有効な要素が存在する場合、その要素の値は引き継がなければならない。

\begin{lstlisting}[language={C++}]
int main()
{
    // 要素数3
    std::vector<int> v = {1,2,3} ;
    // 1万個の要素を保持できるだけのメモリーを予約
    v.reserve( 10000 ) ;
    // vは{1,2,3}
}
\end{lstlisting}

つまり動的メモリー確保をしたあとに、既存の要素を新しいストレージにコピーしなければならないということだ。

まとめよう。

\begin{enumerate}
\def\labelenumi{\arabic{enumi}.}
\item
  すでに指定された要素数以上に予約されているなら何もしない
\item
  まだ動的メモリー確保が行われていなければ動的メモリー確保をする
\item
  有効な要素がある場合は新しいストレージにコピーする
\end{enumerate}

古いストレージから新しいストレージに要素をコピーするとき、古いストレージと新しいストレージが一時的に同時に存在しなければならない。

疑似コード風に記述すると以下のようになる。

\ifTombow\pagebreak\fi
\begin{lstlisting}[language={C++}]
template < typename T >
void f()
{
    // すでに動的確保した古いストレージ
    auto old_ptr = new T ;

    // いま構築した新しいストレージ
    auto new_ptr = new T ;
    // 古いストレージから新しいストレージにコピー
    // *new_ptr = *old_ptr ;
    // 古いストレージを解放
    delete old_value ;
}
\end{lstlisting}

このとき、\texttt{T}型がコピーの最中に例外を投げると、後続の\texttt{delete}が実行されなくなる。この問題に対処して例外安全にするために、C++20に入る見込みの標準ライブラリ、\texttt{std::scope\_exit}\index{scope\_exit@\texttt{scope\_exit}}を使う。

\begin{lstlisting}[language={C++}]
template < typename T >
void f()
{
    // すでに動的確保した古いストレージ
    auto old_ptr = new T ;

    // いま構築した新しいストレージ
    auto new_ptr = new T ;

    // 関数fを抜けるときに古いストレージを解放する。
    std::scope_exit e( [&]{ delete old_ptr ; } ) ;

    // 古いストレージから新しいストレージにコピー
    // *new_ptr = *old_ptr ;

    // 変数eの破棄に伴って古いストレージが解放される
}
\end{lstlisting}

これを踏まえて\texttt{reserve}を実装する。

\begin{lstlisting}[language={C++}]
void reserve( size_type sz )
{
    // すでに指定された要素数以上に予約されているなら何もしない
    if ( sz <= capacity() )
        return ;

    // 動的メモリー確保をする
    auto ptr = allocate( sz ) ;

    // 古いストレージの情報を保存
    auto old_first = first ;
    auto old_last = last ;
    auto old_capacity = capacity() ;

    // 新しいストレージに差し替え
    first = ptr ;
    last = first ;
    reserved_last = first + sz ;

    // 例外安全のため
    // 関数を抜けるときに古いストレージを破棄する
    std::scope_exit e( [&]{
        traits::deallocate( alloc, old_first, old_capacity  ) ;
    } ) ;

    // 古いストレージから新しいストレージに要素をコピー構築
    // 実際にはムーブ構築
    for ( auto old_iter = old_first ; old_iter != old_last ; ++old_iter, ++last )
    {
        // このコピーの理解にはムーブセマンティクスの理解が必要
        construct( last, std::move(*old_iter) ) ;
    }

    // 新しいストレージにコピーし終えたので
    // 古いストレージの値は破棄
    for (   auto riter = reverse_iterator(old_last), rend = reverse_iterator(old_first) ;
            riter != rend ; ++riter )
    {
        destroy( &*riter ) ;
    }
    // scope_exitによって自動的にストレージが破棄される
}
\end{lstlisting}

ここではまだ学んでいないムーブの概念が出てくる。これはムーブセマンティクスの章で詳しく学ぶ。

\clearpage
\hypersection{ch3109}{resize}
\index{resize@\texttt{resize}}\index{vector@\texttt{vector}!resize@\texttt{resize}}

\texttt{resize(sz)}は要素数を\texttt{sz}個にする。

\begin{lstlisting}[language={C++}]
int main()
{
    // 要素数0
    std::vector<int> v ;
    // 要素数10
    v.resize(10) ;
    // 要素数5
    v.resize(5)
    // 要素数変わらず
    v.resize(5)
}
\end{lstlisting}

\texttt{resize}は呼び出し前より要素数を増やすことも減らすこともある。また変わらないこともある。

要素数が増える場合、増えた要素数の値はデフォルト構築された値になる。

\begin{lstlisting}[language={C++}]
struct X
{
    X() { std::cout << "default constructed.\n" ; }
} ;

int main()
{
    std::vector<X> v ;
    v.resize(5) ;
}
\end{lstlisting}

このプログラムを実行すると、\texttt{"default constructed.{\textbackslash}n"}\,は5回標準出力される。

\texttt{resize(sz, value)}は\texttt{resize}を呼び出した結果要素が増える場合、その要素を\texttt{value}で初期化する。

\begin{lstlisting}[language={C++}]
int main()
{
    std::vector<int> v = {1,2,3} ;
    v.resize(5, 4) ;
    // vは{1,2,3,4,4} 
}
\end{lstlisting}

要素数が減る場合、要素は末尾から順番に破棄されていく。

\ifTombow\pagebreak\fi
\begin{lstlisting}[language={C++}]
struct X
{
    ~X()
    { std::cout << "destructed.\n"s ; }
} ;

int main()
{
    std::vector<X> v(5) ;
    v.resize(2) ;
    std::cout << "resized.\n"s ;
}
\end{lstlisting}

このプログラムを実行すると、以下のように出力される。

\begin{lstlisting}[style=terminal]
destructed.
destructed.
destructed.
resized.
destructed.
destructed.
\end{lstlisting}

最初の\texttt{v.resize(2)}で、\texttt{v[4], v[3], v[2]}が書いた順番で破棄されていく。\texttt{main}関数を抜けるときに残りの\texttt{v[1], v[0]}が破棄される。

\texttt{resize(sz)}を呼び出したときに\texttt{sz}が現在の要素数と等しい場合は何もしない。

\begin{lstlisting}[language={C++}]
int main()
{
    // 要素数5
    std::vector<int> v(5) ;
    v.resize(5) ; // 何もしない   
}
\end{lstlisting}

まとめると\texttt{resize}は以下のように動作する。

\begin{enumerate}
\def\labelenumi{\arabic{enumi}.}
\item
  現在の要素数より少なくリサイズする場合、末尾から要素を破棄する
\item
  現在の要素数より大きくリサイズする場合、末尾に要素を追加する
\item
  現在の要素数と等しくリサイズする場合、何もしない
\end{enumerate}

\ifTombow\pagebreak\fi
実装しよう。

\begin{lstlisting}[language={C++}]
    void resize( size_type sz )
    {
        // 現在の要素数より少ない
        if ( sz < size() )
        {
            auto diff = size() - sz ;
            destroy_until( rbegin() + diff ) ;
            last = first + sz ;
        }
        // 現在の要素数より大きい
        else if ( sz > size() )
        {
            reserve( sz ) ;
            for ( ; last != reserved_last ; ++last )
            {
                construct( last ) ;
            }
        }
    }
\end{lstlisting}

要素を破棄する場合、破棄する要素数だけ末尾から順番に破棄する。

要素を増やす場合、\texttt{reserve}を呼び出してメモリーを予約してから、追加の要素を構築する。

\texttt{sz == size()}の場合は、どちらの\texttt{if}文の条件にも引っかからないので、何もしない。

\texttt{size(sz, value)}は、追加の引数を取るほか、\texttt{construct(iter)}の部分が\texttt{constrcut(iter, value)}に変わるだけだ。

\begin{lstlisting}[language={C++}]
void resize( size_type sz, const_reference value )
{
    // ...
            construct( iter, value ) ;
    // ...
}
\end{lstlisting}

これで自作の\texttt{vector}はある程度使えるようになった。コンストラクターで要素数を指定できるし、リサイズもできる。

\begin{lstlisting}[language={C++}]
int main()
{
    vector<int> v(10, 1) ;
    v[2] = 99 ;
    v.resize(5) ;
    // vは{1,1,99,1,1}
}
\end{lstlisting}

\hypersection{ch3110}{push\texttt{\_}back}
\index{push\_back@\texttt{push\_back}}\index{vector@\texttt{vector}!push\_back@\texttt{push\_back}}

\texttt{push\_back}は\texttt{vector}の末尾に要素を追加する。

\begin{lstlisting}[language={C++}]
int main()
{
    std::vector<int> v ;
    // vは{}
    v.push_back(1) ;
    // vは{1}
    v.push_back(2) ;
    // vは{1,2}
}
\end{lstlisting}

\texttt{push\_back}の実装は、末尾の予約された未使用のストレージに値を構築する。もし予約された未使用のストレージがない場合は、新しく動的メモリー確保する。

追加の動的メモリー確保なしで保持できる要素の個数はすでに実装した\texttt{capacity()}で取得できる。\texttt{push\_back}は要素を1つ追加するので、\texttt{size() + 1 <= capacity()}ならば追加の動的メモリー確保はいらない。逆に、\texttt{size() + 1 > capacity()}ならば追加の動的メモリー確保をしなければならない。追加の動的メモリー確保はすでに実装した\texttt{reserve}を使えばよい。

\begin{lstlisting}[language={C++}]
void push_back( const_reference value ) 
{
    // 予約メモリーが足りなければ拡張
    if ( size() + 1 > capacity() )
    {
        // 1つだけ増やす
        reserve( size() + 1 ) ;
    }

    // 要素を末尾に追加
    construct( last, value ) ;
    // 有効な要素数を更新
    ++last ;
}
\end{lstlisting}

これは動く。ただし、効率的ではない。自作の\texttt{vector}を使った以下のような例を見てみよう。

\begin{lstlisting}[language={C++}]
int main()
{
    // 要素数10000
    vector<int> v(1000) ;
(@\ifTombow\pagebreak\fi@)
    // 10001個分のメモリーを確保する
    // 10000個の既存の要素をコピーする
    v.push_back(0) ;
    // 10002個分のメモリーを確保する
    // 10001個の既存の要素をコピーする
    v.push_back(0) ;
}
\end{lstlisting}

たった1つの要素を追加するのに、毎回動的メモリー確保と既存の全要素のコピーをしている。これは無駄だ。

\texttt{std::vector}は\texttt{push\_back}で動的メモリー確保が必要な場合、\texttt{size()+1}よりも多くメモリーを確保する。こうすると、\texttt{push\_back}を呼び出すたびに毎回動的メモリー確保と全要素のコピーを行う必要がなくなるので、効率的になる。

ではどのくらい増やせばいいのか。10個ずつ増やす戦略は以下のようになる。

\begin{lstlisting}[language={C++}]
void push_back( const_reference value ) 
{
    // 予約メモリーが足りなければ拡張
    if ( size() + 1 > capacity() )
    {
        // 10個増やす
        reserve( capacity() + 10 ) ;
    }
    construct( last, value ) ;
    ++last ;
}
\end{lstlisting}

しかしこの場合、以下のようなコードで効率が悪い。

\begin{lstlisting}[language={C++}]
int main()
{
    std::vector<int> v ;
    for ( auto i = 0 ; i != 10000 ; ++i )
    {
        v.push_back(i) ;
    }
}
\end{lstlisting}

10個ずつ増やす戦略では、この場合に1000回の動的メモリー確保と全要素のコピーが発生する。

上のような場合、\texttt{vector}の利用者が事前に\texttt{v.reserve(10000)}とすれば効率的になる。しかし、コンパイル時に要素数がわからない場合、その手も使えない。

\ifTombow\pagebreak\fi
\begin{lstlisting}[language={C++}]
int main()
{
    std::vector<int> inputs ;
    // 要素数は実行時にしかわからない
    // 10万個の入力が行われるかもしれない
    std::copy(
        std::ostream_iterator<int>(std::cin),
        std::ostream_iterator<int>(),
        std::back_inserter(inputs) ) ;
}
\end{lstlisting}

よくある実装は、現在のストレージサイズの2倍のストレージを確保する戦略だ。

\begin{lstlisting}[language={C++}]
void push_back( const_reference value ) 
{
    // 予約メモリーが足りなければ拡張
    if ( size() + 1 > capacity() )
    {
        // 現在のストレージサイズ
        auto c = size() ;
        // 0の場合は1に
        if ( c == 0 )
            c = 1 ;
        else
            // それ以外の場合は2倍する
            c *= 2 ;

        reserve( c ) ;
    }
    construct( last, value ) ;
    ++last ;
}
\end{lstlisting}

\texttt{size()}は\texttt{0}を返す場合もあるということに注意。単に\texttt{reserve(size()*2)}としたのでは\texttt{size() == 0}のときに動かない。

\hypersubsection{ch311001}{shrink\texttt{\_}to\texttt{\_}fit}
\index{shrink\_to\_fit@\texttt{shrink\_to\_fit}}\index{vector@\texttt{vector}!shrink\_to\_fit@\texttt{shrink\_to\_fit}}

\texttt{shrink\_to\_fit()}は\texttt{vector}が予約しているメモリーのサイズを実サイズに近づけるメンバー関数だ。

本書で実装してきた自作の\texttt{vector}は、\texttt{push\_back}時に予約しているメモリーがなければ、現在の要素数の2倍のメモリーを予約する実装だった。すると以下のようなコードで、
\ifTombow\pagebreak\fi
\begin{lstlisting}[language={C++}]
int main()
{
    vector<int> v ;
    std::copy( std::istream_iterator<int>(std::cin), std::istream_iterator<int>(),
        std::back_inserter(v) ) ;
}
\end{lstlisting}
ユーザーが4万個の\texttt{int}型の値を入力した場合、65536個の\texttt{int}型の値を保持できるだけのメモリーが確保されてしまい、差し引き\texttt{sizeof(int) * 25536}バイトのメモリーが未使用のまま確保され続けてしまう。

メモリー要件の厳しい環境ではこのようなメモリーの浪費を避けたい。しかし、実行時にユーザーから任意の個数の入力を受けるプログラムを書く場合には、\texttt{push\_back}を使いたい。

こういうとき、\texttt{shrink\_to\_fit}は\texttt{vector}が予約するメモリーを切り詰めて実サイズに近くする、かもしれない。「かもしれない」というのは、C++の標準規格は\texttt{shrink\_to\_fit}が必ずメモリーの予約サイズを切り詰めるよう規定してはいないからだ。

自作の\texttt{vector}では必ず切り詰める実装にしてみよう。

まず予約するメモリーを切り詰めるとはどういうことか。現在予約しているメモリーで保持できる最大の要素数は\texttt{capacity()}で得られる。実際に保持している要素数を返すのは\texttt{size()}だ。すると\texttt{size() == capacity()}になるようにすればいい。

\begin{lstlisting}[language={C++}]
vector<int> v ;
// ...
v.shrink_to_fit() ;
v.size() == v.capacity() ; // trueにする
\end{lstlisting}

\texttt{shrink\_to\_fit()}を呼んだとき、すでに\texttt{size() == capacity()}が\texttt{true}である場合は、何もしなくてもよい。

それ以外の場合は、現在の有効な要素数分の新しいストレージを確保し、現在の値を新しいストレージにコピーし、古いメモリーは破棄する。

\begin{lstlisting}[language={C++}]
void shrink_to_fit()
{
    // 何もする必要がない
    if ( size() == capacity() )
        return ;

    // 新しいストレージを確保
    auto ptr = allocate( size() ) ;
    // コピー
    auto current_size = size() ;
    for (   auto raw_ptr = ptr, iter = begin(), iter_end = end() ;
            iter != iter_end ; ++iter, ++raw_ptr )
    {
        construct( raw_ptr, *iter ) ;
    }
    // 破棄
    clear() ;
    deallocate() ;
    // 新しいストレージを使う
    first = ptr ;
    last = ptr + current_size ;
    reserved_last = last ;
}
\end{lstlisting}

この実装は\texttt{reserve}と似ている。

\hypersection{ch3111}{vectorのその他のコンストラクター}

\hypersubsection{ch311101}{イテレーターのペア}

\texttt{std::vector}はイテレーターのペアを取り、その参照する値で要素を初期化できる。

\begin{lstlisting}[language={C++}]
int main()
{
    std::array<int, 5> a {1,2,3,4,5} ;
    std::vector<int> v( std::begin(a), std::end(a) ) ;
    // vは{1,2,3,4,5}
}
\end{lstlisting}

これはすでに実装したメンバー関数を使えば簡単に実装できる。

\begin{lstlisting}[language={C++}]
template < typename InputIterator >
vector( InputIterator first, InputIterator last, const Allocator & = Allocator() )
{
    reserve( std::distance( first, last ) ;
    for ( auto i = first ; i != last ; ++i )
    {
        push_back( *i ) ;
    }
}
\end{lstlisting}

\clearpage
\hypersubsection{ch311102}{初期化リスト}

\texttt{std::vector}は配列のように初期化できる。

\begin{lstlisting}[language={C++}]
int main()
{
    std::vector<int> v = {1,2,3} ;
}
\end{lstlisting}

このような初期化を\textsf{リスト初期化}と呼ぶ。

リスト初期化に対応するためには、\texttt{std::initializer\_list<T>}\,を引数に取るコンストラクターを追加する。

 \ifTombow\enlargethispage{3mm}\fi
\begin{lstlisting}[language={C++}]
template < typename T, Allocator = std::allocator<T> >
{
// コンストラクター
vector( std::initializer_list<value_type> init, const Allocator & = Allocator() ) ; 
    // 省略...
} ;
\end{lstlisting}

\texttt{std::initializer\_list<T>}\,は\texttt{T}型の要素を格納する標準ライブラリで、\texttt{\{a,b,c,...\}}\,のようなリスト初期化で構築することができる。

\begin{lstlisting}[language={C++}]
std::initializer_list<int> init = {1,2,3,4,5} ;
\end{lstlisting}

\texttt{std::initializer\_list}は\texttt{begin}/\texttt{end}によるイテレーターを提供しているので、すでに実装したコンストラクターにデリゲートすればよい。

\begin{lstlisting}[language={C++}]
vector( std::initializer_list<value_type> init, const Allocator & alloc = Allocator() ) ; 
    : vector( std::begin(init), std::end(init), alloc )
{ }
\end{lstlisting}
