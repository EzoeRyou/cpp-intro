\hyperchapter{ch21}{プログラマーの三大美徳}{プログラマーの三大美徳}

プログラミング言語Perlの作者、Larry Wallは著書『プログラミングPerl』の初版で以下のように宣言した。
\index{Wall, Larry}

\begin{quote}
読者はプログラマーの三大美徳である、怠惰、短気、傲慢を会得すべきである。
\end{quote}
\index{ぷろぐらまのさんだいびとく@プログラマーの三大美徳}

第2版の巻末の用語集では、以下のような定義が与えらた。

\begin{description}
\item[怠惰]
プログラマーは労力を削減するための労力を惜しまないこと。怠惰のために書いたプログラムは他人にも便利であり、そしてドキュメントを書くことにより自ら他人の質問に答えずに済むようにすること。これがプログラマーの第一の美徳である。これが本書の書かれた理由である。
\item[短気]
コンピューターが怠惰であるときにプログラマーが感ずる怒り。短気によって書かれたプログラムは、単に労力を削減するばかりではなく、事前に解決しておく。少なくとも、すでに解決済みのように振る舞う。これがプログラマーの第二の美徳である。
\item[傲慢]
ゼウスも罰したもう過剰なまでの驕り。他人がそしりを入れられぬほどのプログラムを書く推進剤。これがプログラマーの第三の美徳である。
\end{description}

これから学ぶ\texttt{array}を実装するためのC++の機能を学ぶときに、このプログラマーの三大美徳のことを頭に入れておこう。
