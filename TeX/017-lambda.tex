\hyperchapter{ch17}{ラムダ式}{ラムダ式}

実は以下の形の関数は、「関数」\index{かんすう@関数}ではない。

\begin{lstlisting}[language={C++}]
auto function = []( auto value ) { return value } ;
\end{lstlisting}

これは\texttt{ラムダ式}\index{らむだしき@ラムダ式}と呼ばれるC++の機能で、関数のように振る舞うオブジェクトを作るための式だ。

\hypersection{ch1701}{基本}

\texttt{ラムダ式}の基本の文法は以下のとおり。

\begin{lstlisting}[style=grammar]
[](){} ;
\end{lstlisting}

これを細かく分解すると以下のようになる。

\begin{lstlisting}[style=grammar]
[]  // ラムダ導入子
()  // 引数リスト
{}  // 複合文
;   // 文末
\end{lstlisting}

\texttt{ラムダ導入子}はさておく。

\texttt{引数リスト}\index{ひきすうりすと@引数リスト}\index{らむだしき@ラムダ式!ひきすうりすと@引数リスト}は通常の関数と同じように型名と名前を書ける。

\begin{lstlisting}[language={C++}]
void f( int x, double d ) { }

[]( int x, double d ) { } ;
\end{lstlisting}

\texttt{ラムダ式}では、\texttt{引数リスト}に\texttt{auto}キーワードが使える。

\begin{lstlisting}[language={C++}]
[]( auto x ) { } ;
\end{lstlisting}

このように書くとどんな型でも受け取れるようになる。

\begin{lstlisting}[language={C++}]
int main()
{
    auto f = []( auto x )
    { std::cout << x ; } ;

    f(0) ; // int
    f(1.0) ; // double
    f("hello"s) ; // std::string
}
\end{lstlisting}

\texttt{複合文}\index{ふくごうぶん@複合文}\index{らむだしき@ラムダ式!ふくごうぶん@複合文}は\,\texttt{\{\}}\,\index{\{\}@\texttt{\{\}}}だ。この\,\texttt{\{\}}\,の中に通常の関数と同じように複数の文を書くことができる。

\begin{lstlisting}[language={C++}]
[]()
{
    std::cout << "hello"s ;
    int x = 1 + 1 ;
} ;
\end{lstlisting}

最後の\texttt{文末}\index{ぶんまつ@文末}\index{らむだしき@ラムダ式!ぶんまつ@文末}は\texttt{文}の最後に付けるセミコロンだ。これは\,\texttt{"1+1 ;"}\,とするのと変わらない。\texttt{"1+1"}\,や\,\texttt{"[]()\{\}"}\,は\texttt{式}で、\texttt{文}は\texttt{式}を使うことができる。\texttt{式}だけが入った\texttt{文}を専門用語では\texttt{式文}と呼ぶが特に覚える必要はない。

\begin{lstlisting}[language={C++}]
1 + 1 ; // OK、式文
[](){} ; // OK、式文
\end{lstlisting}

\texttt{ラムダ式}は\texttt{式}なので\texttt{式文}の中に書くことができる。

\texttt{ラムダ式}は\texttt{式}なので、そのまま\texttt{関数呼び出し}することもできる。

\begin{lstlisting}[language={C++}]
void f( std::string x )
{
    std::cout << x ;
}

int main()
{
    f( "hello"s ) ;
    []( auto x ){ std::cout << x ; }( "hello"s ) ;
}
\end{lstlisting}

これはわかりやすくインデントすると以下のようになる。

\begin{lstlisting}[language={C++}]
f               // 関数
( "hello"s ) ;  // 関数呼び出し

// ラムダ式
[]( auto x ){ std::cout << x ; }
( "hello"s ) ;  // 関数呼び出し
\end{lstlisting}

ラムダ式が引数を1つも取らない場合、\texttt{引数リスト}は省略できる。

\begin{lstlisting}[language={C++}]
// 引数を取らないラムダ式
[](){} ;
// 引数リストは省略できる
[]{} ;
\end{lstlisting}

ラムダ式の戻り値の型は\texttt{return}文\index{return@\texttt{return}文}から推定される。

\begin{lstlisting}[language={C++}]
// int
[]{ return 0 ; } ;
// double
[]{ return 0.0 ; } ;
// std::string
[]{ return "hello"s ; } ;
\end{lstlisting}

\texttt{return}文で複数の型を返した場合は推定ができないのでエラーになる。

\begin{lstlisting}[language={C++}]
[]( bool b )
{
    if ( b )
        return 0 ;
    else
        return 0.0 ;
} ;
\end{lstlisting}

戻り値の型を指定したい場合は\texttt{引数リスト}のあとに\,\texttt{->}\,\index{->@\texttt{->}}を書き、型名を書く。

\begin{lstlisting}[language={C++}]
[]( bool b ) -> int
{
    if ( b )
        return 0 ;
    else
        // doubleからintへの変換
        return 0.0 ;
} ;
\end{lstlisting}

戻り値の型の推定は通常の関数も同じだ。

\begin{lstlisting}[language={C++}]
// int
auto f() { return 0 ; }

// 戻り値の型の明示的な指定
auto f() -> int { return 0 ; }
\end{lstlisting}

\hypersection{ch1702}{キャプチャー}
\index{きやぷちや@キャプチャー}\index{らむだしき@ラムダ式!きやぷちや@キャプチャー}

\texttt{ラムダ式}は書かれている関数のローカル変数を使うことができる。これを\texttt{キャプチャー}という。\texttt{キャプチャー}は通常の関数にはできない\texttt{ラムダ式}の機能だ。

\begin{lstlisting}[language={C++}]
void f()
{
    // ローカル関数
    auto message = "hello"s ;

    [=](){ std::cout << message ; } ;
}
\end{lstlisting}

\texttt{キャプチャー}には\texttt{コピーキャプチャー}と\texttt{リファレンスキャプチャー}がある。

\hypersubsection{ch170201}{コピーキャプチャー}
\index{こぴきやぷちや@コピーキャプチャー}\index{らむだしき@ラムダ式!こぴきやぷちや@コピーキャプチャー}

\texttt{コピーキャプチャー}は変数をコピーによってキャプチャーする。

\texttt{コピーキャプチャー}をするには、\texttt{ラムダ式}を\texttt{[=]}\index{[=]@\texttt{[=]}}\index{らむだしき@ラムダ式![=]@\texttt{[=]}}と書く。

\begin{lstlisting}[language={C++}]
int main()
{
    int x = 0 ;
    // コピーキャプチャー
    [=]{ return x ; } ;
}
\end{lstlisting}

\texttt{コピーキャプチャー}した変数はラムダ式の中で変更できない。

\begin{lstlisting}[language={C++}]
int main()
{
    int x = 0 ;
    // エラー
    [=]{ x = 0 ; } ;
}
\end{lstlisting}

変更できるようにする方法もあるのだが、通常は使われない。

\hypersubsection{ch170202}{リファレンスキャプチャー}
\index{りふあれんすきやぷちや@リファレンスキャプチャー}\index{らむだしき@ラムダ式!りふあれんすきやぷちや@リファレンスキャプチャー}

\texttt{リファレンスキャプチャー}は変数をリファレンスによってキャプチャーする。

\texttt{リファレンス}を覚えているだろうか。リファレンスは初期化時の元の変数を参照する変数だ。

\begin{lstlisting}[language={C++}]
int main()
{
    int x = 0 ;
    // 通常の変数
    int y = x ;

    // 変数を変更
    y = 1 ;
    // xの値は変わらない

    // リファレンス
    int & ref = x ;

    // リファレンスを変更
    ref = 1 ;
    // xの値が変わる
}
\end{lstlisting}

\texttt{リファレンスキャプチャー}を使うには、\texttt{ラムダ式}を\texttt{[\&]}\index{[\&]@\texttt{[\&]}}\index{らむだしき@ラムダ式![\&]@\texttt{[\&]}}と書く。

\begin{lstlisting}[language={C++}]
int main()
{
    int x = 0 ;
    [&] { return x ; } ;
}
\end{lstlisting}

\texttt{リファレンスキャプチャー}した変数を\texttt{ラムダ式}の中で変更すると、元の変数が変更される。

\begin{lstlisting}[language={C++}]
int main()
{
    int x = 0 ;
    auto f = [&]{ ++x ; } ;

    f() ; // x == 1
    f() ; // x == 2
    f() ; // x == 3
}
\end{lstlisting}

ラムダ式についてはまだいろいろな機能があるが、本書での解説はここまでとする。
