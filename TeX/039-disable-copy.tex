\hypersection{ch3504}{コピーの禁止}
\index{こぴ@コピー!きんし@禁止}

型によっては、コピーという概念が存在しないものがある。

例えばコピー不可能なシステムのリソースを扱うクラスだ。

具体的にはファイル、スレッド、プロセス、ネットワークソケットといったリソースだ。このようなリソースを管理するクラスを作ったとして、いったいコピーをどうすればいいのだろうか。

コピーできないクラスは\texttt{deleted定義}\index{deletedていぎ@\texttt{deleted}定義}を使ってコピーコンストラクターとコピー代入演算子を消すことができる。

\texttt{deleted定義}は関数の本体\,\texttt{\{...\}}\,の代わりに\,\texttt{= delete}\index{= delete@\texttt{= delete}}を書く。\texttt{deleted定義}されている関数を使うとエラーとなる。

\begin{lstlisting}[language={C++}]
struct X
{
    // コピーコンストラクター
    X( const X & ) = delete ;
    // コピー代入演算子
    X & operator = ( const X & ) = delete ;

    // デフォルトコンストラクター
    X() { }
    // ムーブコンストラクター
    X ( X && ) { }
    // ムーブ代入演算子
    X & operator = ( X && ) { }
} ;
\end{lstlisting}

このようなクラス\texttt{X}は、コピーできない。

\begin{lstlisting}[language={C++}]
int main()
{
    // デフォルト構築できる
    X a ;
    // エラー、 コピーできない
    X b = a ;
    b = a ;
    // OK、ムーブはできる。
    X c = std::move(a) ;
}
\end{lstlisting}

クラス\texttt{X}はコピーコンスラクターとコピー代入演算子が\texttt{deleted定義}されているために、コピーをすることができない。

コピーやムーブが禁止されている型をデータメンバーに持つクラスは、デフォルトのコピーやムーブができなくなる。

\begin{lstlisting}[language={C++}]
// コピーできない型
struct Uncopyable
{
    Uncopyable(){}
    Uncopyable( const Uncopyable & ) = delete ;
    Uncopyable & operator = ( const Uncopyable & ) = delete ; 
} ;

// デフォルトのコピーができない
struct X
{
    Uncopyable member ;
} ;
\end{lstlisting}

\texttt{deleted定義}を使えばムーブも禁止できる。ただし、ムーブを禁止するというのは現実的にはあまり実用性がない。というのも、コピーはムーブでもあるので、コピーを提供している型はムーブとしてコピーを行えばムーブも提供できることになる。

\hypersection{ch3505}{5原則}

C++には「5原則」\index{5げんそく@5原則}\index{C++!5げんそく@5原則}という作法がある。

5原則とは、
\begin{enumerate}
\def\labelenumi{\arabic{enumi}.}
\item
  コピーコンストラクター
\item
  コピー代入演算子
\item
  ムーブコンストラクター
\item
  ムーブ代入演算子
\item
  デストラクター
\end{enumerate}

\hspace{2zw}このうちの1つを独自に定義したならば、残りの4つも定義すべきである。

\vskip 1.0zw
\noindent
というものだ。

なぜか。コピーやムーブを独自に定義するということは、デフォルトのコピーやムーブでは足りない何らかの処理をしたいはずだ。その処理には、たいていの場合何らかの破棄の処理が必要で、するとデストラクターも定義しなければならない。

同様に、デストラクターで何らかの独自の処理をするということは、コピーやムーブでも何らかの処理をしたいはずだ。
