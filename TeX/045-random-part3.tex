\hypersection{ch3913}{正規分布}

\hypersubsection{ch391301}{正規分布(std::normal\texttt{\_}distribution\texttt{<}RealType\texttt{>})}
\index{せいきぶんぷ@正規分布}\index{らんすうぶんぷ@乱数分布!せいきぶんぷ@正規分布}

\texttt{std::normal\_distribution<RealType>}\,\index{normal\_distribution@\texttt{normal\_distribution}}は浮動小数点数型の乱数\(x\)を以下の確率密度関数に従って分布する。
\[
 (x\,|\,\mu,\sigma)
      = \frac{1}{\sigma \sqrt{2\pi}}
        \cdot
        % e^{-(x-\mu)^2 / (2\sigma^2)}
        \exp{\left(- \, \frac{(x - \mu)^2}
                             {2 \sigma^2}
             \right)
            }
 \text{ .}
\]

分布パラメーターのうちの\(\mu\)と\(\sigma\)は、それぞれ分布の平均(mean)、標準偏差(standard deviation)とも呼ばれている。

\vskip 1.0zw
\noindent
\textsf{変数の宣言:}

\begin{lstlisting}[style=grammar]
std::normal_distribution<RealType> d( mean, stddev ) ;
\end{lstlisting}

\texttt{RealType}は浮動小数点数型でデフォルトは\texttt{double}。\texttt{mean}, \texttt{stddev}は浮動小数点数型。\texttt{mean}は平均。\texttt{stddev}は標準偏差で値の範囲は\(0 < stddev\)。

\vskip 1.0zw
\noindent
\textsf{使い方:}

\begin{lstlisting}[language={C++}]
int main()
{
    std::mt19937 e ;
    std::normal_distribution d( 0.0, 1.0 ) ;
    d.mean() ; // 0.0
    d.stddev() ; // 1.0

    for ( int i = 0 ; i != 10 ; ++i )
    {
        std::cout << d(e) << ", "sv ;
    }  
}
\end{lstlisting}

\hypersubsection{ch391302}{対数正規分布(std::lognormal\texttt{\_}distribution\texttt{<}RealType\texttt{>})}
\index{たいすうせいきぶんぷ@対数正規分布}\index{らんすうぶんぷ@乱数分布!たいすうせいきぶんぷ@対数正規分布}

\texttt{std::lognormal\_distribution<RealType>}\,\index{lognormal\_distribution@\texttt{lognormal\_distribution}}は浮動小数点数の乱数\(x > 0\)を以下の確率密度関数に従って分布する。
\[
p(x\,|\,m,s) = \frac{1}{s x \sqrt{2 \pi}}
     \cdot \exp{\left(-\frac{(\ln{x} - m)^2}{2 s^2}\right)}
     \text{ .}
\]

\vskip 1.0zw
\noindent
\textsf{変数の宣言:}

\begin{lstlisting}[style=grammar]
std::lognormal_distribution<RealType> d( m, s ) ;
\end{lstlisting}

\texttt{RealType}は浮動小数点数型でデフォルトは\texttt{double}。\texttt{m}, \texttt{s}は\texttt{RealType}型。値の範囲は\(0 < s\)。

\vskip 1.0zw
\noindent
\textsf{使い方:}

\begin{lstlisting}[language={C++}]
int main()
{
    std::mt19937 e ;
    std::lognormal_distribution d( 0.0, 1.0 ) ;
    d.m() ; // 0.0
    d.s() ; // 1.0

    for ( int i = 0 ; i != 10 ; ++i )
    {
        std::cout << d(e) << ", "sv ;
    }  
}
\end{lstlisting}

\hypersubsection{ch391303}{カイ二乗分布(std::chi\texttt{\_}squared\texttt{\_}distribution\texttt{<}RealType\texttt{>})}
\index{かいじじようぶんぷ@カイ二乗分布}\index{らんすうぶんぷ@乱数分布!かいじじようぶんぷ@カイ二乗分布}

\texttt{std::chi\_squared\_distribution<RealType>}\,\index{chi\_squared\_distribution@\texttt{chi\_squared\_distribution}}は浮動小数点数型の乱数\(x > 0\)を以下の確率密度関数に従って分布する。
\[
p(x\,|\,n) = \frac{x^{(n/2)-1} \cdot e^{-x/2}}{\Gamma(n/2) \cdot 2^{n/2}} \text{ .} 
\]

\ifTombow\pagebreak\else{\vskip 1.0zw}\fi
\noindent
\textsf{変数の宣言:}

\begin{lstlisting}[style=grammar]
std::chi_squared_distribution<RealType> d( n ) ;
\end{lstlisting}

\texttt{RealType}は浮動小数点数型でデフォルトは\texttt{double}。\texttt{n}は\texttt{RealType}型。値の範囲は\(0 < n\)。

\vskip 1.0zw
\noindent
\textsf{使い方:}

\begin{lstlisting}[language={C++}]
int main()
{
    std::mt19937 e ;
    std::chi_squared_distribution d( 1.0 ) ;
    d.n() ; // 1.0

    for ( int i = 0 ; i != 10 ; ++i )
    {
        std::cout << d(e) << ", "sv ;
    }  
}
\end{lstlisting}

\hypersubsection{ch391304}{コーシー分布(std::cauchy\texttt{\_}distribution\texttt{<}RealType\texttt{>})}
\index{こしぶんぷ@コーシー分布}\index{らんすうぶんぷ@乱数分布!こしぶんぷ@コーシー分布}

\texttt{std::cauchy\_distribution<RealType>}\,\index{cauchy\_distribution@\texttt{cauchy\_distribution}}は浮動小数点数型の乱数\(x\)を以下の確率密度関数に従って分布する。
\[
p(x\,|\,a,b) = \left(\pi b \left(1 + \left(\frac{x-a}{b} \right)^2 \, \right)\right)^{-1} \text{ .} 
\]

\vskip 1.0zw
\noindent
\textsf{変数の宣言:}

\begin{lstlisting}[style=grammar]
std::cauchy_distribution<RealType> d( a, b ) ;
\end{lstlisting}

\texttt{RealType}は浮動小数点数型でデフォルトは\texttt{double}。\texttt{a}, \texttt{b}は\texttt{RealType}型。値の範囲は\(0 < b\)。

\vskip 1.0zw
\noindent
\textsf{使い方:}

\begin{lstlisting}[language={C++}]
int main()
{
    std::mt19937 e ;
    std::chi_squared_distribution d( 0.0, 1.0 ) ;
    d.a() ; // 0.0
    d.b() ; // 1.0

    for ( int i = 0 ; i != 10 ; ++i )
    {
        std::cout << d(e) << ", "sv ;
    }  
}
\end{lstlisting}

\hypersubsection{ch391305}{フィッシャーの\(F\)分布(std::fisher\texttt{\_}f\texttt{\_}distribution\texttt{<}RealType\texttt{>})}
\index{ふいつしやのFぶんぷ@フィッシャーの\(F\)分布}\index{らんすうぶんぷ@乱数分布!ふいつしやのFぶんぷ@フィッシャーの\(F\)分布}

フィッシャーの\(F\)分布(Fisher's \(F\)-distribution)の名前は数学者サー・ロナルド・エイルマー・フィッシャー(Sir Ronald Aylmer Fisher)\index{Fisher, Ronald Aylmer}に由来する。

\texttt{std::fisher\_f\_distribution<RealType>}\,\index{fisher\_f\_distribution@\texttt{fisher\_f\_distribution}}は浮動小数点数の乱数\(x > 0\)を以下の関数密度関数に従って分布する。
\[
p(x\,|\,m,n) = \frac{\Gamma\big((m+n)/2\big)}{\Gamma(m/2) \; \Gamma(n/2)}
     \cdot \left(\frac{m}{n}\right)^{m/2}
     \cdot x^{(m/2)-1}
     \cdot \left(1 + \frac{m x}{n}\right)^{-(m + n)/2}
     \text{ .}
\]

\vskip 1.0zw
\noindent
\textsf{変数の宣言:}

\begin{lstlisting}[style=grammar]
std::fisher_f_distribution<RealType> d( m, n ) ;
\end{lstlisting}

\texttt{RealType}は浮動小数点数型でデフォルトは\texttt{dobule}。\texttt{m}, \texttt{n}は\texttt{RealType}型。値の範囲は\(0 < m\) かつ \(0 < n\)。

\vskip 1.0zw
\noindent
\textsf{使い方:}

\begin{lstlisting}[language={C++}]
int main()
{
    std::fisher_f_distribution d( 1.0 ) ;
    d.n() ; // 1.0

    std::mt19937 e ;
    d(e) ;
}
\end{lstlisting}

\hypersubsection{ch391306}{スチューデントの\(t\)分布(std::student\texttt{\_}t\texttt{\_}distribution\texttt{<}RealType\texttt{>})}
\index{すちゆでんとのtぶんぷ@スチューデントの\(t\)分布}\index{らんすうぶんぷ@乱数分布!すちゆでんとのtぶんぷ@スチューデントの\(t\)分布}

スチューデントの\(t\)分布(Student's \(t\)-distribution)はウィリアム・シーリー・ゴセット(William Sealy Gosset)によって考案された。当時、ウィリアムはギネス醸造所で働いていたが、ギネスは従業員に科学論文を発表することを禁じていたために、ウィリアムはスチューデントという偽名で発表した。

\texttt{std::student\_t\_distribution<RealType>}\,\index{student\_t\_distribution@\texttt{student\_t\_distribution}}は浮動小数点数型の乱数\(x\)を以下の確率密度関数に従って分布する。
\[
p(x\,|\,n) = \frac{1}{\sqrt{n \pi}}
     \cdot \frac{\Gamma\big((n+1)/2\big)}{\Gamma(n/2)}
     \cdot \left(1 + \frac{x^2}{n} \right)^{-(n+1)/2}
     \text{ .}
\]

\ifTombow\pagebreak\else{\vskip 1.0zw}\fi
\noindent
\textsf{変数の宣言:}

\begin{lstlisting}[style=grammar]
std::student_t_distribution<RealType> d( n ) ;
\end{lstlisting}

\texttt{RealType}は浮動小数点数型でデフォルトは\texttt{double}。\texttt{n}は\texttt{RealType}型で、値の範囲は\(0 < n\)。

\vskip 1.0zw
\noindent
\textsf{使い方:}

\begin{lstlisting}[language={C++}]
int main()
{
    std::student_t_distribution d( 1.0 ) ;
    d.n() ; // 1.0

    std::mt19937 e ;
    d(e) ;
}
\end{lstlisting}

